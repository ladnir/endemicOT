\section{All But One OT from Key Agreement}\label{sec:endemicOT}




\begin{figure}[h!]
\centering
\begin{tikzpicture}
\node [anchor=west] at (-1,5) {Sender:};
\node [anchor=west] at (7,5) {Receiver$(\i\in[n])$:};
\node [anchor=west] at (7,4.5){$r_{i}\leftarrow\G$};
\node [anchor=west] at (7,4){for $j=1$ to $n-1$};
\node [anchor=west] at (7,3.5){$\tape_{\A,i+j\bmod n}\leftarrow\bits^*$};
\node [anchor=west] at (7,3){$\mes_\A\leftarrow \UKA.\A(\tape_\A)$};
\node [anchor=west] at (7,2.5){$r_{\i}:=\mes_\A\ominus\H((r_j)_{j\in[n]\setminus\{\i\}})$};
\draw [thick, <-] (4.75,3)-- node [midway,above]{$(r_j)_{j\in[n]}$}(6.75,3);
\node [anchor=west] at (-1,3){$\forall j\in[n]:$};
\node [anchor=west] at (-1,2.5){$\quad\mes_{\A,j}:=r_j\oplus\H((r_\ell)_{\ell\in[n]\setminus\{j\}})$};
\node [anchor=west] at (-1,2){$\quad\tape_{\B,j}\leftarrow\bits^*$};
\node [anchor=west] at (-1,1.5){$\quad\mes_{\B,j}\leftarrow\UKA.\B(\tape_{\B,j},\mes_{\A,j})$};
\draw [thick, ->] (4.75,1.5)-- node [midway,above]{$(\mes_{\B,j})_{j\in[n]}$}(6.75,1.5);
\node [anchor=west] at (7,0.5) {$\key_{\A,\i}:=\UKA.\EK(\tape_\A,\mes_{\B,\i})$};
\node [anchor=west] at (-1,1) {$\forall j\in[n]$};
\node [anchor=west] at (-1,0.5) {$\quad\key_{\B,j}:=\UKA.\EK(\tape_{\B,j},\mes_{\A,j})$};
\end{tikzpicture}
\caption{The figure depicts a $1$ out of $n$ oblivious transfer using as a building block a \UKA and a random oracle $\H:\G^{n-1}\rightarrow\G$, where \G is a group with group operations $\oplus$, $\ominus$. By the correctness of the \UKA scheme, $\key_{\A,\i}=\key_{\B,\i}$ holds.}
\label{fig:KAtoOT}
\end{figure}

\begin{figure}[h!]
\centering
\begin{tikzpicture}
\node [anchor=west] at (-1,5) {Sender:};
\node [anchor=west] at (7,5) {Receiver$(\i\in[n])$:};
\node [anchor=west] at (7,4.5){$\forall j\in[n]\setminus\{\i\}: r_{j}\leftarrow\G$};
\node [anchor=west] at (7,4){$\tape_\A\leftarrow\bits^*$};
\node [anchor=west] at (7,3.5){$\mes_\A\leftarrow \UKA.\A(\tape_\A)$};
\node [anchor=west] at (7,3){$r_{\i}:=\mes_\A\ominus\H((r_j)_{j\in[n]\setminus\{\i\}})$};
\draw [thick, ->] (4.75,3.15)-- node [midway,above]{$\mes_{\B}$}(6.75,3.15);
\draw [thick, <-] (4.75,3)-- node [midway,below]{$(r_j)_{j\in[n]}$}(6.75,3);
\node [anchor=west] at (-1,3){$\forall j\in[n]:$};
\node [anchor=west] at (-1,2.5){$\quad\mes_{\A,j}:=r_j\oplus\H((r_\ell)_{\ell\in[n]\setminus\{j\}})$};
\node [anchor=west] at (-1,4.5){$\tape_{\B}\leftarrow\bits^*$};
\node [anchor=west] at (-1,4){$\mes_{\B}\leftarrow\UKA.\B(\tape_{\B})$};
\node [anchor=west] at (7,2) {$\key_{\A,\i}:=\UKA.\EK(\tape_\A,\mes_{\B})$};
\node [anchor=west] at (-1,2) {$\quad\key_{\B,j}:=\UKA.\EK(\tape_{\B},\mes_{\A,j})$};
\end{tikzpicture}
\caption{The figure depicts a $1$ out of $n$ oblivious transfer using as a building block a one-round \UKA and a random oracle $\H:\G^{n-1}\rightarrow\G$, where \G is a group with group operations $\oplus$, $\ominus$. By the correctness of the \UKA scheme, $\key_{\A,\i}=\key_{\B,\i}$ holds.}
\label{fig:oneroundKAtoOT}
\end{figure}


