\section{Additional Definitions and Lemmas}



\begin{definition}[Generalize Ideal $k$-out-of-$n$ Oblivious Transfer]\label{def:got}
	A (generalized) \emph{ideal $k$-out-of-$n$ oblivious transfer} is a functionality that interacts with two parties, a sender \send and a receiver \rec. Let $\set\subseteq[n]$ of size $k$ and  $s_1,...,s_n\in \{0,1\}^\ell$.
	
	The functionality is publicly parameterized by one of the following message sampling methods:
	\begin{description}
		\item[] \textsc{Sender Chosen Message:} $\send$ sends the circuit $\mathcal{M} : [n] \rightarrow \{0,1\}^\ell$ to the functionality which defines $s_i:=\mathcal{M}(i)$.
		
		\item[] \textsc{Receiver Chosen Message:} \rec sends the circuit  $\mathcal{M} : [k] \rightarrow \{0,1\}^\ell$ to the functionality which defines $s_{\set_i}:=\mathcal{M}(i)$ for $i\in[k]$ and uniformly samples $s_i\gets\{0,1\}^\ell$ for $i\in\set\setminus[n]$.
		
		\item[] \textsc{Uniform Message:} The functionality uniformly samples $s_i\gets\{0,1\}^{\ell}$ for $i\in[n]$. 
		
		\item[] \textsc{Endemic Chosen Message:} If $\send$ is corrupt, then $\send$ sends the circuit $\mathcal{M} : [n] \rightarrow \{0,1\}^\ell$ to the functionality which defines $s_i:=\mathcal{M}(i)$.
		
		If \rec is corrupt, \rec sends the circuit  $\mathcal{M} : [k] \rightarrow \{0,1\}^\ell$ to the functionality which defines $s_{\set_i}:=\mathcal{M}(i)$ for $i\in[k]$.
		
		All remaining $s_i$ for $i\in [n]$ are uniformly samples $s_i\gets\{0,1\}^\ell$.
	\end{description}
	
	The functionality is publicly parameterized by one of the following selection methods:
	\begin{description}
		\item[] \textsc{Receiver Selection:} \rec sends the circuit $\mathcal{S}:[n]\rightarrow\{0,1\}$ to the functionality where the support of $\mathcal{S}$ is of size $k$. The functionality defines $\set:=\{i \mid \mathcal{S}(i)=1\}$.
		\item[] \textsc{Uniform Selection:} The functionality uniformly samples $\set\gets\mathbb{P}([n])$ s.t. $|\set|=k$.
	\end{description}
	
	As specified by the message sampling method, the oracle receives the circuit $\mathcal{M}$ from the appropriate party if one is called for.  As specified by the selection method, the functionality receives the circuit $\mathcal{S}$ if one is called for. 
	Thereafter, upon receiving the message $(\textsc{Output}, i)$ from \send, respond with $s_i$. Upon receiving $(\textsc{Output}, i)$ from \rec and if $i\in \set$,  respond with $s_i$. 
	
	We denote the ideal functionalities for Sender Chosen, Receiver Chosen, Uniform and Endemic Chosen Message with Receiver Selection as $\OOT^\send, \OOT^\rec,\OOT^\U,\OOT^\E$, respectively. The analogous oracles for Uniform Selection are denoted as  $\OOT^{\send u},$ $\OOT^{\rec u}, \OOT^{\U u}, \OOT^{\E u}$, respectively.
\end{definition}
\begin{remark}
	When $n$ is polynomial in the security parameter $\kappa$, we simplify the above definition to \definitionref{def:ot} to allow the parties directly input the appropriate $s_i$ messages as opposed to specifying a circuit $\mathcal{M}$. Similarly for the set $\set:=\{i \mid \mathcal{S}(i)=1\}$. Lastly, instead of querying the oracle with $(\textsc{Output}, i)$, the oracle sends $(s_i)_{i\in [n]}$ to \send and $(\set, (s_i)_{i\in\set})$ to \rec. This simplification can trivially be simulated when $n=\textsf{poly}(\kappa)$.
\end{remark}



\begin{definition}[Coin Tossing]\label{def:coin}
An \emph{ideal coin tossing} is a functionality denoted with $\F^{\coin}$ that interacts with two parties \A and \B, samples a uniform string $r\in\bits^*$ and sends $r$ to $\A$ and $\B$.
\end{definition}


\begin{figure}
\centering
\framebox{
\begin{tikzpicture}
\node [anchor=west] at (0,5) {Sender:};
\node [anchor=west] at (7.5,5) {Receiver:};
\draw [thick] (4.75,4.65)--(5.75,4.65)--(5.75,3.65)--(4.75,3.65)--cycle;
\node at (5.25,4.125){$\OOT^\send$};
\node [anchor=west] at (7.5,4.5){$\set'\gets \mathbb{P}([n])$ s.t. $|\set'|=k$};
\draw [thick, <-] (5.75,4.5)-- node [midway,above]{$\set'$}(7.25,4.5);
\draw [thick, ->] (5.75,3.85)-- node [midway,above]{$(s_{\pi(i)})_{i\in \set'}$}(7.25,3.85);
\node [anchor=west] at (0,4.5) {$\pi\leftarrow\Pi([n])$};
\draw [thick, ->] (3.25,3)-- node [midway, above]{$\pi$} (7.25,3);
\draw [thick, ->] (3.25,4.5)-- node [midway,above]{$(s_{\pi(i)})_{i\in [n]}$}(4.75,4.5);
\node [anchor=west] at (0,2.5) {output $(s_{i})_{i\in[n]}$};
\node [anchor=west] at (7.5,3) {$\set:=\{j \mid \exists i, j=\pi(i)\}$};
\node [anchor=west] at (7.5,2.5) {output $\set, (s_i)_{i\in\set}$};
\end{tikzpicture}
}
% 	\framebox{\begin{minipage}{0.95\linewidth}
% 			Input: The sender \send and receiver \rec have no input.
% 			\begin{enumerate}
% 				\item \label{step:uniformSelect_step1} \rec uniformly samples $\set'\gets \mathbb{P}([n])$ s.t. $|\set'|=k$. \rec sends $\set'$ to $\OOT^{\E}$ and receives $(s_i')_{i\in \set'}$ in response. \send receives $(s_i')_{i\in [n]}$. 
% 				\item \label{step:uniformSelect_step2} \send samples a random permutation $\pi : [n] \rightarrow [n]$ and sends it to \rec.
% 				\item \label{step:uniformSelect_step3} \send outputs $(s_i)_{i\in [n]}$ and \rec outputs $(\set, (s_i)_{i\in\set})$ where $s_i:=s_{\pi(i)}'$ and $\set:=\{j \mid \exists i, j=\pi(i)\}$.
% 			\end{enumerate}
% 	\end{minipage}}
	\caption{Uniform Selection $k$-out-of-$n$ OT protocol $\Pi^{\send u}$ in the $\OOT^{\send}$ hybrid. $\Pi([n])$ is the set of permutations over $[n]$.}
	\label{fig:uniformSelect}
\end{figure}


\begin{lemma}
	$\Pi^{\E u}$ of \figureref{fig:uniformSelect} realizes the  Ideal Uniform Selection Endemic Message OT $\OOT^{\send u}$ (\definitionref{def:ot}) with unconditional security in the $\OOT^{\send}$ hybrid.
\end{lemma}

\begin{remark}
	The same transformation applies to $\OOT^\U,\OOT^\E, \OOT^\rec$ except \send does not input anything to $\OOT^*$.
\end{remark}
 

\begin{proof}[sketch]
	Consider a corrupt \send. Due to $\set'$ being uniformly distributed, so is $\set=\pi(\set')$ since $\pi$ is one to one. Consider a corrupt \rec. The simulator receives $\set'$ from \rec and the $\set,(s_i)_{i\in\set}$ from $\OOT^{\send u}$. The simulator uniformly samples $\pi$ s.t. $\{s_i\}_{i\in\set} =\{s_{\pi(i)}\}_{i\in\set'}$ and completes the protocol.
	\pe
\end{proof}