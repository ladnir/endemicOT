\section{Additional Preliminary Definitions and Lemmata}\label{sec:defsandlems}



\begin{definition}[Coin Tossing]\label{def:coin}
An \emph{ideal coin tossing} is a functionality denoted with $\F^{\coin}$ that interacts with two parties \A and \B, samples a uniform string $r\in\bits^*$ and sends $r$ to $\A$ and $\B$.
\end{definition}

\newcommand{\extract}{\ensuremath{\mathsf{ext}}\xspace}

\begin{definition}[Extractable Commitments]\label{def:com}
An \emph{extractable commitment scheme} consists of three algorithms.
\begin{description}
\item[$\com(x,r)$:] Commits to $x$ using randomness $r$. 
\item[$\open(\com,x,r)$:] Outputs $1$ if commitment $\com\in\com(x,r)$.
\item[$\extract(\com,\aux)$:] Given some auxiliary information, it extracts committed value $x$. 
\end{description}
For security, we ask that it is hiding, i.e. for any $x,m$, $x, \com(x,r)$ is indistinguishable from $m,\com(x,r)$ and that it is binding, i.e. for any $x$, $\extract(\com(x,r))$ outputs $x$.
\end{definition}

An extractable commitment can easily constructed using a random oracle by defining $\com(x,r):=H(x,r)$, $\open$ simply evaluates $\H$ and checks equality and the $\extract$ algorithm observes the random oracle queries from which $x$ can be learned.

\subsection{Key Agreement}\label{sec:addKA}
We give the following additional security definition for key agreement protocols.

\begin{definition}[One-Round Uniform Key Agreement]
We call a \UKA \emph{one-round uniform key agreement} if the function $\mes_{\B}\leftarrow\B(\tape_{\B},\mes_\A)$ does not depend on $\mes_A$ and can be computed solely using input $\tape_{\B}$. More precisely, there is a function $\B'$ such that for any $\mes_{\A}$, $\B'(\tape_{\B})=\B(\tape_{\B},\mes_{\A})$, which we will in the following refer to with $\B$ as well. 
\end{definition}

\begin{definition}[Multi-Instance Uniformity]
We call a \UKA $Q$-\emph{multi-instance $\epsilon$-uniform} if for any ppt distinguisher \D and any polynomial size auxiliary input $z$,
$$
|\Pr[\D^{\O_{\A}}(z)]=1]-Pr[\D^{\O_{u}}(z)=1]|\leq \epsilon,
$$
where $\O_{\A}$ outputs $\mes_\A\leftarrow\A(\tape_\A)$ for fresh randomness $\tape_\A$ and $\O_u$ outputs $u\leftarrow\G$ and $Q$ is a bound on the amount of queries to $\O_{u}$, $\O_{\A}$.
\end{definition}



\begin{definition}[Multi-Instance Key-Indistinguishability]
We call a \UKA $(Q,n)$-\emph{multi-instance $\epsilon$-key-indistinguishable} if for any ppt distinguisher \D and any polynomial size auxiliary input $z$,
$$
|\Pr[\D^{\O_{\langle \A,\B\rangle}, \O_{\key}}(z)]=1]-Pr[\D^{\O_{\langle \A,\B\rangle},\O_{u}}(z)=1]|\leq \epsilon,
$$
where $\O_{\langle \A,\B\rangle}$ outputs on the $i$-th query a transcript $\T_i:=\langle \A_i,\B_i\rangle$, $\O_{\key}$ outputs on query $j$, key $\key_j=\EK(\tape_{\A,j},\mes_{\B,j})=\EK(\tape_{\B,j},\mes_{\A,j})$ that matches transcript $\T_i$. $\O_u$ outputs a uniform element $u$ from the key domain. $\O_{\langle \A,\B\rangle}$ uses fresh random tapes $\tape_{\A,i},\tape_{\B,i}\leftarrow\bits^*$ for every query. $Q$ is a bound on the amount of queries to $\O_{\langle \A,\B\rangle}$, where $n$ bounds the amount of queries to $\O_{\key}$, $\O_{u}$. 
\end{definition}

In case of a one-round \UKA, we define a stronger version of the multi-instance key-indistinguishability, which we call one-round multi-instance key-indistinguishability.

\begin{definition}[One-Round Multi-Instance Key-Indistinguishability]
We call a one-round \UKA \emph{one-round $(Q,n)$-multi-instance $\epsilon$-key-indistinguishability} if for any ppt distinguisher \D and any polynomial size auxiliary input $z$,
$$
|\Pr[\D^{\O_{\A}, \O_{\key}}(z,\mes_\B)]=1]-Pr[\D^{\O_{\A},\O_{u}}(z,\mes_\B)=1]|\leq\epsilon,
$$
where $\mes_{\B}\leftarrow\B(\tape_{\B})$ for uniform $\tape_{\B}$. $\O_{\A}$ outputs on the $i$-th query $\mes_{\A,i}\leftarrow \A(\tape_{\A,i})$ for uniform $\tape_{\A,i}$. $\O_{\key}$ outputs on query $j$, key $\key_j=\EK(\tape_{\A,j},\mes_{\B})=\EK(\tape_{\B},\mes_{\A})$. $\O_u$ outputs a uniform element $u$ from the key domain. $Q$ is a bound on the amount of queries to $\O_{\A}$, where $n$ bounds the amount of queries to $\O_{\key}$, $\O_{u}$. 
\end{definition}

In the next lemmata, we show that all the security notions are implied by the standard definitions, but potentially with a polynomial security loss.
 
\begin{lemma}\label{lem:multuniform}
Let \UKA be $\epsilon$-uniform, then it is $Q$-multi-instance $Q\epsilon$-uniform.
\end{lemma}
\begin{proof}
This follows straightforwardly from using a simple hybrid argument. Hybrid $\hyb_{i}$ samples $\mes_{\A,j}$ for $j\leq i$ from $\O_u$ and for $j>i$ from $\O_{\A}$. If there is an adversary that distinguishes $\hyb_i$ from $\hyb_{i+1}$ for any $i$, then we can break the uniformity of \UKA by distinguishing $\mes_{\A,i+1}$ from uniform.  
\pe
\end{proof}

\begin{lemma}\label{lem:keytomultkey}
Let \UKA be $\epsilon$-key-indistinguishable, then it is $(Q,n)$-multi-instance $Q\epsilon$-key-indistinguishable. 
\end{lemma}

\begin{proof}
Again, we use a hybrid argument over hybrids $\hyb_i$. In $\hyb_i$, $(\mes_{\A,j},\mes_{\B,j}),\key_j$ is sampled from $\O_{\langle \A,\B\rangle}\times\O_u$ for $j\leq i$ and from $\O_{\langle \A,\B\rangle}\times\O_\key$ for $j>i$. If one distinguishes $\hyb_i$ from $\hyb_{i+1}$ for some $i$, one breaks the key-indistinguishability. 
\pe
\end{proof}

\begin{lemma}\label{lem:oneroundkeytomultkey}
Let a one-round \UKA be $\epsilon$-key-indistinguishable, then it is one-round $(Q,n)$-multi-instance $Q\epsilon$-key-indistinguishable. 
\end{lemma}

\begin{proof}
This lemma follows for the same reason as \lemmaref{lem:keytomultkey}.
\pe
\end{proof}


\subsection{Oblivious Transfer}\label{sec:addOT}


\begin{lemma}[Repeat of \lemmaref{lemma:is_a}]\label{lemma:is_a_repeat}
	Let the distribution of OT strings be efficiently sampleable. 
	Then $\OOT^\U$-security implies $\OOT^{\send}$ as well as $\OOT^\rec$-security. $\OOT^{\send}$ or $\OOT^\rec$-security imply $\OOT^\E$-security.
\end{lemma}

\begin{proof}
	In the first step, we show that uniform message security implies sender chosen message security and receiver chosen message security implies endemic security. These two implications result from the same simple fact that a malicious sender interacting with the ideal OT is easier to construct when it can choose the OT strings than when it receives the strings from the ideal OT. The following claim formalizes this fact. 
	\begin{claim}\label{claim:utocs}
		Let $\Pi$ be an OT secure against a malicious sender with respect to an ideal OT $\OOT^*$ that sends the OT strings $(s_i)_{i\in[n]}$ to the sender, i.e. functionality $\OOT^\U$ and $\OOT^{\rec}$, and the distribution of $(s_i)_{i\in[n]}$ is efficiently sampleable. Then $\Pi$ is also secure against a malicious sender with respect to ideal OT $\OOT$, which receives the OT strings $(s_i)_{i\in[n]}$ from the sender, i.e functionality $\OOT^\send$ and $\OOT^\E$.
	\end{claim}
	
	
	\begin{proof}
		We show that if there is an adversary that breaks the security against a malicious security with respect to ideal OT $\OOT$ then there is also an adversary that breaks the security with respect to $\OOT^*$. More precisely, if there is a ppt adversary $\Adv_1$ such that for any ppt adversary $\Adv_1'$ there exists a ppt distinguisher $\D_1$ with 
		$$
		|\Pr[\D_1((\Adv_1,\rec)_{\Pi})=1] -\Pr[\D_1( (\Adv'_1, \OOT))=1]|=\epsilon,
		$$
		where all algorithms receive input $1^\sec$ and \rec additionally receives input \set.
		Then there is also a ppt adversary $\Adv_2$ such that for any ppt adversary $\Adv_2'$ there exists a ppt distinguisher $\D_2$ with 
		$$
		|\Pr[\D_2((\Adv_2,\rec)_{\Pi})=1] -\Pr[\D_1( (\Adv'_2, \OOT^*))=1]|=\epsilon,
		$$
		where all algorithms receive input $1^\sec$ and \rec additionally receives input \set.
		
		We set $\Adv_2:=\Adv_1$ and $\D_2:=\D_1$. Further, for any $\Adv_2'$, there is an $\Adv_1'$  such that the distribution of $(\Adv'_2, \OOT^*)$ is identical with the distribution $(\Adv'_1, \OOT)$. This follows from the fact that  $\Adv_1'$ could choose the OT strings $(s_i)_{i\in[n]}$  from the same distribution as $\OOT^*$ does and otherwise follow the description of $\Adv_2'$. Since $\D_1$ is successful for any $\Adv_1'$ it will be also for any $\Adv_2'$, which can be seen as a subset of the set of all ppt adversaries $\Adv_1'$.
		\pe
	\end{proof}
	
	The remaining two implications, from uniform security to receiver chosen message security and from sender chosen message security to endemic security follow in a similar fashion. Again it is easier to construct a malicious receiver interacting with the ideal OT when he can choose the OT strings rather than receiving them from the ideal OT.
	\begin{claim}\label{claim:utocr}
		Let $\Pi$ be an OT secure against a malicious receiver with respect to an ideal OT $\OOT^*$ that sends the learned OT strings $(s_i)_{i\in\set}$ to the receiver, i.e. functionality $\OOT^\U$ and $\OOT^{\send}$, and the distribution of $(s_i)_{i\in\set}$ is efficiently sampleable. Then $\Pi$ is also secure against a malicious sender with respect to ideal OT $\OOT$, which receives the OT strings $(s_i)_{i\in\set}$ from the receiver, i.e. $\OOT^\rec$ and $\OOT^\E$.
	\end{claim}
	
	\begin{proof}
		The proof is basically identical to the proof of Claim~\ref{claim:utocs}. Again, the set of all ppt $\Adv_2'$ is a subset of the set of all ppt $\Adv_1'$ and identical with the set of all $\Adv_1'$ that sample  $(s_i)_{i\in\set}$ from the same distribution as when sent by $\OOT^*$.
		\pe
	\end{proof}
	\pe
\end{proof}



In the following, we give a generalized definition of OT.

\begin{definition}[Generalize Ideal $k$-out-of-$n$ Oblivious Transfer]\label{def:got}
	A (generalized) \emph{ideal $k$-out-of-$n$ oblivious transfer} is a functionality that interacts with two parties, a sender \send and a receiver \rec. Let $\set\subseteq[n]$ of size $k$ and  $s_1,...,s_n\in \{0,1\}^\ell$.
	
	The functionality is publicly parameterized by one of the following message sampling methods:
	\begin{description}
		\item[] \textsc{Sender Chosen Message:} $\send$ sends the circuit $\mathcal{M} : [n] \rightarrow \{0,1\}^\ell$ to the functionality which defines $s_i:=\mathcal{M}(i)$.
		
		\item[] \textsc{Receiver Chosen Message:} \rec sends the circuit  $\mathcal{M} : [k] \rightarrow \{0,1\}^\ell$ to the functionality which defines $s_{\set_i}:=\mathcal{M}(i)$ for $i\in[k]$ and uniformly samples $s_i\gets\{0,1\}^\ell$ for $i\in\set\setminus[n]$.
		
		\item[] \textsc{Uniform Message:} The functionality uniformly samples $s_i\gets\{0,1\}^{\ell}$ for $i\in[n]$. 
		
		\item[] \textsc{Endemic:} If $\send$ is corrupt, then $\send$ sends the circuit $\mathcal{M} : [n] \rightarrow \{0,1\}^\ell$ to the functionality which defines $s_i:=\mathcal{M}(i)$.
		
		If \rec is corrupt, \rec sends the circuit  $\mathcal{M} : [k] \rightarrow \{0,1\}^\ell$ to the functionality which defines $s_{\set_i}:=\mathcal{M}(i)$ for $i\in[k]$.
		
		All remaining $s_i$ for $i\in [n]$ are uniformly samples $s_i\gets\{0,1\}^\ell$.
	\end{description}
	
	The functionality is publicly parameterized by one of the following selection methods:
	\begin{description}
		\item[] \textsc{Receiver Selection:} \rec sends the circuit $\mathcal{S}:[n]\rightarrow\{0,1\}$ to the functionality where the support of $\mathcal{S}$ is of size $k$. The functionality defines $\set:=\{i \mid \mathcal{S}(i)=1\}$.
		\item[] \textsc{Uniform Selection:} The functionality uniformly samples $\set\gets\mathbb{P}([n])$ s.t. $|\set|=k$.
	\end{description}
	
	As specified by the message sampling method, the oracle receives the circuit $\mathcal{M}$ from the appropriate party if one is called for.  As specified by the selection method, the functionality receives the circuit $\mathcal{S}$ if one is called for. 
	Thereafter, upon receiving the message $(\textsc{Output}, i)$ from \send, respond with $s_i$. Upon receiving $(\textsc{Output}, i)$ from \rec and if $i\in \set$,  respond with $s_i$. 
	
	We denote the ideal functionalities for sender chosen, receiver chosen, uniform and endemic with receiver selection as $\OOT^\send, \OOT^\rec,\OOT^\U,\OOT^\E$, respectively. The analogous oracles for Uniform Selection are denoted as  $\OOT^{\send u},$ $\OOT^{\rec u}, \OOT^{\U u}, \OOT^{\E u}$, respectively.
\end{definition}
\begin{remark}
	When $n$ is polynomial in the security parameter $\kappa$, we simplify the above definition to \definitionref{def:ot} to allow the parties directly input the appropriate $s_i$ messages as opposed to specifying a circuit $\mathcal{M}$. Similarly for the set $\set:=\{i \mid \mathcal{S}(i)=1\}$. Lastly, instead of querying the oracle with $(\textsc{Output}, i)$, the oracle sends $(s_i)_{i\in [n]}$ to \send and $(\set, (s_i)_{i\in\set})$ to \rec. This simplification can trivially be simulated when $n=\textsf{poly}(\kappa)$.
\end{remark}


The following transformation allows to transform an OT where the receiver's choice bit is chosen to an OT with a random choice bit. This transformation is very useful in the context of OT extension.

\begin{figure}
\centering
\framebox{
\begin{tikzpicture}[xscale=1.4]
\node [anchor=west] at (0,5) {Sender:};
\node [anchor=west] at (5.1,5) {Receiver:};
\begin{scope}[shift={(-2,0)}]
\draw [thick] (4.75,4.65)--(5.75,4.65)--(5.75,3.65)--(4.75,3.65)--cycle;
\node at (5.25,4.125){$\OOT^\send$};
\draw [thick, <-] (5.75,4.5)-- node [midway,above]{$\set'$}(7.1,4.5);
\draw [thick, ->] (5.75,3.85)-- node [midway,above]{$(s_{\pi(i)})_{i\in \set'}$}(7.1,3.85);
\draw [thick, ->] (3.35,3)-- node [midway, above]{$\pi$} (7.1,3);
\draw [thick, ->] (3.35,4.5)-- node [midway,above]{$(s_{\pi(i)})_{i\in [n]}$}(4.75,4.5);
\end{scope}

\node [anchor=west] at (0,4.5) {$\pi\leftarrow\Pi_n$};
\node [anchor=west] at (5.1,4.5){$\set'\gets \mathbb{P}([n])$ s.t. $|\set'|=k$};
\node [anchor=west] at (0,2.5) {$(s_{i})_{i\in[n]}$};
\node [anchor=west] at (5.1,3) {$\set:=\{j \mid \exists i, j=\pi(i)\}$};
\node [anchor=west] at (5.1,2.5) {$\set, (s_i)_{i\in\set}$};
\end{tikzpicture}
}
% 	\framebox{\begin{minipage}{0.95\linewidth}
% 			Input: The sender \send and receiver \rec have no input.
% 			\begin{enumerate}
% 				\item \label{step:uniformSelect_step1} \rec uniformly samples $\set'\gets \mathbb{P}([n])$ s.t. $|\set'|=k$. \rec sends $\set'$ to $\OOT^{\E}$ and receives $(s_i')_{i\in \set'}$ in response. \send receives $(s_i')_{i\in [n]}$. 
% 				\item \label{step:uniformSelect_step2} \send samples a random permutation $\pi : [n] \rightarrow [n]$ and sends it to \rec.
% 				\item \label{step:uniformSelect_step3} \send outputs $(s_i)_{i\in [n]}$ and \rec outputs $(\set, (s_i)_{i\in\set})$ where $s_i:=s_{\pi(i)}'$ and $\set:=\{j \mid \exists i, j=\pi(i)\}$.
% 			\end{enumerate}
% 	\end{minipage}}
	\caption{Uniform selection $k$-out-of-$n$ OT protocol $\Pi^{\send u}$ in the $\OOT^{\send}$ hybrid. $\Pi_n$ is the set of permutations over $[n]$.}
	\label{fig:uniformSelect}
\end{figure}



\begin{lemma}
	$\Pi^{\send u}$ of \figureref{fig:uniformSelect} realizes the  ideal uniform selection sender chosen message OT $\OOT^{\send u}$ (\definitionref{def:ot}) with unconditional security in the $\OOT^{\send}$ hybrid.
\end{lemma}

\begin{remark}
	The same transformation applies to $\OOT^\U,\OOT^\E, \OOT^\rec$ except \send does not input anything to $\OOT^*$.
\end{remark}
 

\begin{proof}[sketch]
	Consider a corrupt \send. Due to $\set'$ being uniformly distributed, so is $\set=\pi(\set')$ since $\pi$ is one to one. Consider a corrupt \rec. The simulator receives $\set'$ from \rec and the $\set,(s_i)_{i\in\set}$ from $\OOT^{\send u}$. The simulator uniformly samples $\pi$ s.t. $\{s_i\}_{i\in\set} =\{s_{\pi(i)}\}_{i\in\set'}$ and completes the protocol.
	\pe
\end{proof}

%
%\subsection{Diffie-Hellman Key Exchange}\label{sec:DDH}
%In this subsection we use a group \G of prime order $p$ and $g$ is a generator or \G. For simplicity, we use the bracket notation $\lb a\rb$ to denote $g^a$, in particular $\lb 1\rb:=g$. For vector $\vec{a}=(a_1,\ldots,a_n)$, we use $\lb \vec{a}\rb$ to denote $(\lb a_1\rb,\ldots,\lb a_n\rb)$. We use the same notation, i.e. $\lb A\rb$, also for a matrix $A$.
%In \figureref{fig:DH}, we depict the Diffie-Hellman key exchange. 
%\begin{figure}[h!]
\centering
\framebox{
\begin{minipage}{0.85\linewidth}
\centering
\begin{tikzpicture}
\node [anchor=west] at (2,5) {\A:};
\node [anchor=west] at (7,5) {\B:};
\node [anchor=west] at (2,4.5){$a\leftarrow\Z_p$};
\draw [thick, ->] (3.75,4.5)-- node [midway,above]{$\mes_{\A}:=\lb  a\rb$}(6.75,4.5);
\node [anchor=west] at (7,4.5){$b\leftarrow\Z_p$};
\draw [thick, <-] (3.75,4.35)-- node [midway,below]{$\mes_{\B}:=\lb  b\rb$}(6.75,4.35);
\node [anchor=west] at (2,4) {$\key_\A=\mes_{\B}^a$};
\node [anchor=west] at (7,4) {$\key_\B=\mes_{\A}^b$};
\end{tikzpicture}
\end{minipage}
}
\caption{\label{fig:DH}The figure shows the Diffie-Hellman key exchange. Correctness follows from $\key_\A=\lb  ab\rb=\key_{\B}$.}
\end{figure}
%It can be proven secure under the decisional Diffie-Hellman (DDH) assumption.
%\begin{definition}[Decisional Diffie-Hellman (DDH) Assumption]
%For a group $\G$, the \emph{decisional Diffie-Hellman} assumption is hard if for any ppt distinguisher \D,
%$$
%|\Pr\lb \D(\lb 1\rb,\lb a\rb,\lb b\rb,\lb ab\rb)=1\rb-\Pr\lb D(\lb 1\rb,\lb a\rb,\lb b\rb,\lb c\rb)=1\rb|=\negl,
%$$
%where $a\leftarrow\Z_p$, $b\leftarrow\Z_p$ and $c\leftarrow\Z_p$.
%\end{definition}
%
%The following lemma states that multi-instance DDH is secure under the DDH assumption with a security loss equal to the amount of instances.
%
%\begin{lemma}\label{lem:DHuniform}
%Let DDH be secure over \G except advantage $\epsilon$, then $n$-multi-instance DDH is secure over \G except at most advantage $n\epsilon$.
%\end{lemma}
%
%\begin{proof}
%This lemma follows from a simple hybrid argument. We define hybrid $\hyb_i$ as the distribution over $\lb \vec{a}\rb,\lb b\rb,\lb \vec{c}\rb$, where for all $j\leq i$, $c_j=a_jb$ and for $j>i$, $c_j\leftarrow\Z_p$. If $\D$ distinguishes hybrid $\hyb_i$ from $\hyb_{i+1}$, it breaks the DDH assumption over \G.
%\pe
%\end{proof}
%
