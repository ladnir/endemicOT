\section{Relations between OT Security Notions}\label{sec:relot}

We show now how endemic security relates to the other security notions of uniform, receiver and sender chosen message security. For an overview, see \figureref{fig:OTrelations}.
\begin{lemma}\label{lemma:is_a}
Let the distribution of OT strings be efficiently sampleable. 
Then $\OOT^\U$-security implies $\OOT^{\send}$ as well as $\OOT^\rec$-security. $\OOT^{\send}$ or $\OOT^\rec$-security imply $\OOT^\E$-security.
\end{lemma}

\begin{proof}
In the first step, we show that uniform message security implies sender chosen message security and receiver chosen message security implies endemic security. These two implications result from the same simple fact that a malicious sender interacting with the ideal OT is easier to construct when it can choose the OT strings than when it receives the strings from the ideal OT. The following claim formalizes this fact. 
\begin{claim}\label{claim:utocs}
Let $\Pi$ be an OT secure against a malicious sender with respect to an ideal OT $\OOT^*$ that sends the OT strings $(s_i)_{i\in[n]}$ to the sender, i.e. functionality $\OOT^\U$ and $\OOT^{\rec}$, and the distribution of $(s_i)_{i\in[n]}$ is efficiently sampleable. Then $\Pi$ is also secure against a malicious sender with respect to ideal OT $\OOT$, which receives the OT strings $(s_i)_{i\in[n]}$ from the sender, i.e functionality $\OOT^\send$ and $\OOT^\E$.
\end{claim}


\begin{proof}
We show that if there is an adversary that breaks the security against a malicious security with respect to ideal OT $\OOT$ then there is also an adversary that breaks the security with respect to $\OOT^*$. More precisely, if there is a ppt adversary $\Adv_1$ such that for any ppt adversary $\Adv_1'$ there exists a ppt distinguisher $\D_1$ with 
$$
|\Pr[\D_1((\Adv_1,\rec)_{\Pi})=1] -\Pr[\D_1( (\Adv'_1, \OOT))=1]|=\epsilon,
$$
where all algorithms receive input $1^\sec$ and \rec additionally receives input \set.
Then there is also a ppt adversary $\Adv_2$ such that for any ppt adversary $\Adv_2'$ there exists a ppt distinguisher $\D_2$ with 
$$
|\Pr[\D_2((\Adv_2,\rec)_{\Pi})=1] -\Pr[\D_1( (\Adv'_2, \OOT^*))=1]|=\epsilon,
$$
where all algorithms receive input $1^\sec$ and \rec additionally receives input \set.

We set $\Adv_2:=\Adv_1$ and $\D_2:=\D_1$. Further, for any $\Adv_2'$, there is an $\Adv_1'$  such that the distribution of $(\Adv'_2, \OOT^*)$ is identical with the distribution $(\Adv'_1, \OOT)$. This follows from the fact that  $\Adv_1'$ could choose the OT strings $(s_i)_{i\in[n]}$  from the same distribution as $\OOT^*$ does and otherwise follow the description of $\Adv_2'$. Since $\D_1$ is successful for any $\Adv_1'$ it will be also for any $\Adv_2'$, which can be seen as a subset of the set of all ppt adversaries $\Adv_1'$.
\pe
\end{proof}

The remaining two implications, from uniform security to receiver chosen message security and from sender chosen message security to endemic security follow in a similar fashion. Again it is easier to construct a malicious receiver interacting with the ideal OT when he can choose the OT strings rather than receiving them from the ideal OT.
\begin{claim}\label{claim:utocr}
Let $\Pi$ be an OT secure against a malicious receiver with respect to an ideal OT $\OOT^*$ that sends the learned OT strings $(s_i)_{i\in\set}$ to the receiver, i.e. functionality $\OOT^\U$ and $\OOT^{\send}$, and the distribution of $(s_i)_{i\in\set}$ is efficiently sampleable. Then $\Pi$ is also secure against a malicious sender with respect to ideal OT $\OOT$, which receives the OT strings $(s_i)_{i\in\set}$ from the receiver, i.e. $\OOT^\rec$ and $\OOT^\E$.
\end{claim}

\begin{proof}
The proof is basically identical to the proof of Claim~\ref{claim:utocs}. Again, the set of all ppt $\Adv_2'$ is a subset of the set of all ppt $\Adv_1'$ and identical with the set of all $\Adv_1'$ that sample  $(s_i)_{i\in\set}$ from the same distribution as when sent by $\OOT^*$.
\pe
\end{proof}
\pe
\end{proof}



Even though endemic security is implied by all the other security notions and could be seen as the weakest, it is still sufficient to obtain any of the other notions by using very simple transformations. In the following lemmas we show these transformations and prove their security.   

\begin{figure}[t]
\centering
\framebox{
\begin{tikzpicture}
\node [anchor=west] at (0,5) {Sender$((s_i)_{i\in[n]})$:};
\node [anchor=west] at (7.5,5) {Receiver$(\set\subset[n])$:};
\draw [thick, <-] (5.75,4.5)-- node [midway,above]{$\set$}(7.25,4.5);
\draw [thick] (4.75,4.65)--(5.75,4.65)--(5.75,3.65)--(4.75,3.65)--cycle;
\node at (5.25,4.125){$\OOT^\E$};
\node [anchor=west] at (0,4){$\forall i\in[n]:$};
\node [anchor=west] at (0,3.5){$\quad r_i:=s_i\oplus m_i$};

\draw [thick, ->] (3.25,3)-- node [midway,above]{$(r_i)_{i\in[n]}$}(7.25,3);
\node [anchor=west] at (7.5,3.5){$\forall i\in\set:$};
\node [anchor=west] at (7.5,3){$\quad s_i:=r_i\oplus m_i$};
\draw [thick, ->] (5.75,3.85)-- node [midway,above]{$(m_i)_{i\in \set}$}(7.25,3.85);
\draw [thick, <-] (3.25,3.85)-- node [midway,above]{$(m_i)_{i\in [n]}$}(4.75,3.85);
\node [anchor=west] at (0,2.5) {output $(s_i)_{i\in[n]}$};
\node [anchor=west] at (7.5,2.5) {output $(s_i)_{i\in\set}$};
\end{tikzpicture}
}
% 	\framebox{\begin{minipage}{0.95\linewidth}
% 			Input: The sender $\send$ has input $s_1,...,s_n\in \{0,1\}^\ell$ while the receiver $\rec$ has input $\set\subseteq [n]$ of size $k$.
% 			\begin{enumerate}
% 				\item \label{step:senderChosen_step1} The parties invoke $\O^{e}_{k,n}$ where $\rec$ takes the role of the receiver with inputs $\set$. $\send$ obtains $(m_i)_{i\in[n]}$ and $\rec$ obtains  from  $\O^{e}_{k,n}$. 
% 				\item \label{step:senderChosen_step2}  $\send$ sends the strings $r_1,...,r_n$ to $\rec$ where $r_i:=s_i\oplus m_i$.
% 				\item \label{step:senderChosen_step3}  $\rec$ outputs $(s_i)_{i\in \set}$ where $s_i:=m_i\oplus r_i$.
% 	\end{enumerate}\end{minipage}}
	\caption{Sender Chosen OT Protocol $\Pi^{\send}_{k,n}$ in the $\OOT^\E$ hybrid (\definitionref{def:ot}). For all $i\in[n]$, $r_i$, $m_i$ and $s_i$ are in $\bits^\ell$.}
	\label{fig:protoSendOT}
\end{figure}


\begin{lemma}\label{lem:EtoS}
	$\Pi^{\send}_{k,n}$ of \figureref{fig:protoSendOT} realizes the Sender Chosen Ideal OT $\OOT^{\send}$ (\definitionref{def:ot}) with unconditional security in the $\OOT^\E$ hybrid.
\end{lemma}

\iffullversion
\begin{proof}
 We construct a new adversary $\Adv_1'$ which interacts with functionality $\OOT^\send$ and produces an identical output  to $\Adv_1$.  $\Adv_1'$ plays the role of $\OOT^\E$ and $\rec$ in $\Pi^{\send}_{k,n}$ while running $\Adv_1$. $\Adv_1'$ receives the endemic OT strings $m_1,...,m_n$ from $\Adv_1$%in \stepref{step:senderChosen_step1}
, along with the strings $r_1,...,r_n$% in \stepref{step:senderChosen_step2}
.  $\Adv_1'$ extracts the OT strings of $\Adv_1$ as $s_i:=r_i\oplus m_i$. $\Adv_1'$ sends $(s_i)_{i\in [n]}$ to $\OOT^\send$ and outputs whatever $\Adv_1$ outputs.
	
	Observe that the output of the honest receiver is identical when $\Adv_1$ interacts with $\rec$ in $\Pi^{\send}_{k,n}$ and when $\Adv_1'$ interacts with $\OOT^\send$.
	
	
	
	Now consider a corrupt receiver $\Adv$. We construct a new adversary $\Adv_2'$ which interacts with the functionality $\OOT^\send$ and produces an identical output distribution to $\Adv_2$. $\Adv_2'$ plays the role of $\OOT^\E$ and $\send$ in $\Pi^{\send}_{k,n}$ while running $\Adv_2$. $\Adv_2'$ receives the set $\set\subset[n]$ of size $k$ and the endemic OT strings $(m_i)_{i\in \set}$ from $\Adv_2$. $\Adv_2'$ sends $\set$ to  $\OOT^\send$ and receives $(s_i)_{i\in \set}$ in response. $\Adv_2'$ computes $r_i:=s_i\oplus m_i$ for $i\in \set$ and uniformly samples $r_i\gets \{0,1\}^\ell$ for $i\in [n]\setminus \set$. $\Adv_2'$ sends $r_1,...,r_n$ to $\Adv_2$ and outputs whatever $\Adv_2$ outputs.

	The transcripts of $\Adv_2$ in these two interactions are identical except for $(r_i)_{i\in [n]\setminus \set}$. Observe that in the real interaction for $i\in [n]\setminus\set$, $r_i:=s_i\oplus m_i$, where $m_i$ is sampled uniformly at random  by functionality $\OOT^\E$ and is independent of the transcript of $\Adv_2$ (conditioned on $r_i$). Therefore sampling $r_i$ directly induces an identical distribution. 
	
	Therefore, any distinguishing advantage $\Adv_1$ or $\Adv_2$ produces in protocol $\Pi^{\send}_{k,n}$ implies that $\Adv_1'$ or $\Adv_2'$ would produce the same advantage against the instantiation of $\OOT^\E$, i.e. $\negl$.
\end{proof}
\else
\begin{proof}[sketch]	
	The case of a corrupt sender is straight forward. The simulator receives $(m_i)_{i\in[n]},(r_i)_{i\in[n]}$ from \rec and sends $(m_i\oplus r_i)_{i\in[n]}$ to $\OOT^\send$. The output distribution is identical to the real protocol. Consider a corrupt receiver. The simulator receives their set and strings $(m_i)_{i\in\in{\set}}$. All other $(m_i)_{i\in\in{[n]\setminus\set}}$ are uniformly distributed in their view. Therefore, so is $(m_i+r_i)_{i\in\in{[n]\setminus\set}}$ and the simulator can then replace these strings with $(r_i)_{i\in\in{[n]\setminus\set}}$.
\pe
\end{proof}
\fi

\begin{figure}[t]
\centering
\framebox{
\begin{tikzpicture}
\node [anchor=west] at (0,5) {Sender:};
\draw [thick, <-] (5.75,4.5)-- node [midway,above]{$\set$}(7.25,4.5);
\node [anchor=west] at (7.5,5) {Receiver$(\set\subset[n],(s_i)_{i\in\set})$:};
\draw [thick] (4.75,4.65)--(5.75,4.65)--(5.75,3.65)--(4.75,3.65)--cycle;
\node at (5.25,4.125){$\OOT^\E$};
\node [anchor=west] at (0,3.5){$\forall i\in[n]:$};
\node [anchor=west] at (0,3){$\quad s_i:=r_i\oplus m_i$};

\draw [thick, <-] (3.25,3)-- node [midway,above]{$(r_i)_{i\in[n]}$}(7.25,3);
\node [anchor=west] at (7.5,4.5){$\forall i\in[n]\setminus\set:$};
\node [anchor=west] at (7.5,4){$\quad r_i\leftarrow\bits^\ell$};
\node [anchor=west] at (7.5,3.5){$\forall i\in\set:$};
\node [anchor=west] at (7.5,3){$\quad r_i:=s_i\oplus m_i$};
\draw [thick, ->] (5.75,3.85)-- node [midway,above]{$(m_i)_{i\in \set}$}(7.25,3.85);
\draw [thick, <-] (3.25,3.85)-- node [midway,above]{$(m_i)_{i\in [n]}$}(4.75,3.85);
\node [anchor=west] at (0,2.5) {output $(s_i)_{i\in[n]}$};
\node [anchor=west] at (7.5,2.5) {output $(s_i)_{i\in\set}$};
\end{tikzpicture}
}
% 	\framebox{\begin{minipage}{0.95\linewidth}
% 			Input: The sender $\send$ has no input while the receiver $\rec$ has input $\set\subseteq [n]$ of size $k$ and $(s_i)_{i\in \set}$ where $s_i\in \{0,1\}^\ell$.
% 			\begin{enumerate}
% 				\item \label{step:receiverChosen_step1} The parties invoke $\O^{e}_{k,n}$ where $\rec$ takes the role of the receiver with inputs $\set$. $\send$ obtains $(m_i)_{i\in[n]}$ and $\rec$ obtains $(m_i)_{i\in \set}$ from  $\O^{e}_{k,n}$. 
% 				\item \label{step:receiverChosen_step2}  $\rec$ sends the strings $r_1,...,r_n$ to $\send$ where $r_i:=s_i\oplus m_i$ for $i\in \set$ and $r_i\gets\{0,1\}^\ell$ for $i\in [n]\setminus\set$.
% 				\item \label{step:receiverChosen_step3}  $\send$ outputs $(s_i)_{i\in [n]}$ where $s_i:=m_i\oplus r_i$.
% 	\end{enumerate}\end{minipage}}
	\caption{Receiver Chosen OT Protocol $\Pi^{\rec}_{k,n}$ in the $\OOT^\E$ hybrid (\definitionref{def:ot}). For all $i\in[n]$, $r_i$, $m_i$ and $s_i$ are in $\bits^\ell$.}
	\label{fig:protoRecvOT}
\end{figure}


\begin{lemma}\label{lem:EtoR}
	$\Pi^{\rec}_{k,n}$ of \figureref{fig:protoRecvOT} realizes the Receiver Chosen Ideal OT $\OOT^{\rec}$ (\definitionref{def:ot}) with unconditional security in the $\OOT^\E$ hybrid.
\end{lemma}
\iffullversion
\begin{proof}
	First let us consider any corrupt sender $\Adv_1$. We construct a new adversary $\Adv_1'$ which interacts with $\OOT^{\rec}$ and produces an identical output distribution as $\Adv_1$.  $\Adv_1'$ plays the role of $\OOT^\E$ and $\rec$ in $\Pi^{\rec}_{k,n}$ while running $\Adv_1$. $\Adv_1'$ receives the endemic OT strings $m_1,...,m_n$ from $\Adv_1$% in \stepref{step:receiverChosen_step1}
. $\Adv_1'$ invokes $\OOT^{\rec}$ as the sender and receives $s_1,...,s_n$ in response. $\Adv_1'$ sends $r_1,...,r_n$ to $\Adv_1$ where $r_i:=m_1\oplus s_i$ and 	 outputs whatever $\Adv_1$ outputs.
	
	The transcripts of $\Adv_1$ in these two interactions are identical except for $(r_i)_{i\in [n]\setminus \set}$. Observe that in the real interaction for $i\in [n]\setminus\set$, $r_i\gets\{0,1\}^\ell$ while $\Adv_1'$ chooses $r_i:=m_1\oplus s_i$ where $s_i$ is sampled uniformly at random by $\OOT^{\rec}$. Therefore, computing $r_i:=m_1\oplus s_i$ implies an identically distribution as when $r_i\gets\{0,1\}^\ell$ given that $s_i\gets\{0,1\}^\ell$ is independent of the transcript of $\Adv_1$. 
	
	
	Now let us consider a corrupt receiver $\Adv_2$. We construct a new adversary $\Adv_2'$ which interacts with $\OOT^{\rec}$ and produces an identical output as $\Adv_2$. $\Adv_2'$ plays the role of $\OOT^\E$ and $\send$ in $\Pi^{\rec}_{k,n}$ while running $\Adv_2$. $\Adv_2'$ receives the set $\set\subset[n]$ of size $k$ and the endemic OT strings $(m_i)_{i\in \set}$ from $\Adv_2$% in \stepref{step:receiverChosen_step1}
. $\Adv_2'$ receives $r_1,...,r_n$ from $\Adv_2$ % in \stepref{step:receiverChosen_step2} 
 and extracts $(s_i)_{i\in \set}$ as $s_i:=r_i\oplus m_i$. $\Adv_2'$ sends $\set$ and $(s_i)_{i\in \set}$ to $\OOT^{\rec}$ and outputs whatever $\Adv_2$ outputs.
	
	The output distribution of he honest sender is identical in these two interactions is identical except for $(s_i)_{i\in [n]\setminus \set}$. In the real interaction \send outputs $s_i:=m_i\oplus r_i$ where $m_i$ is sampled uniformly by $\OOT^\E$ and independent of the transcript. Therefore $\OOT^{\rec}$ sampling $s_i\gets\{0,1\}^\ell$ directly is identically distributed.
	
	Therefore, any distinguishing advantage adversary $\Adv_1$ or $\Adv_2$ produces in protocol $\Pi^{\rec}_{k,n}$ implies  that $\Adv_1'$ or $\Adv_2'$ would produce the same advantage against the instantiation of $\OOT^\E$, i.e. $\negl$.
\end{proof}
\else
\begin{proof}[sketch]
	Consider a corrupt sender. First, \send sends $(m_i)_{i\in[n]}$ to the simulator and $\OOT^\rec$ sends $(s_i)_{i\in[n]}$. $k$ of the $s_i$ strings will have a distribution specified by $\rec$ while the other are uniform. For all of them, the simulator computes $r_i:=s_i+m_i$ and sends $r_i$ to \send. For the uniform $s_i$, adding another uniform value does not change the distribution. The other messages have an identical distribution.	
	For a corrupt receiver, the $(m_i)_{i\in[n]\setminus\set}$ are uniformly distributed in their view. Therefore, so is $(m_i+r_i)_{i\in [n]\setminus\set}$. 
\pe
\end{proof}
\fi


\begin{figure}[t]
\centering
\framebox{
\begin{tikzpicture}
\node [anchor=west] at (0,5) {Sender:};
\node [anchor=west] at (7.5,5) {Receiver$(\set\subset[n])$:};
\draw [thick, <-] (5.75,4.5)-- node [midway,above]{$\set$}(7.25,4.5);
\draw [thick] (4.75,4.65)--(5.75,4.65)--(5.75,3.65)--(4.75,3.65)--cycle;
\node at (5.25,4.125){$\OOT^\E$};
\draw [thick, ->] (5.75,3.85)-- node [midway,above]{$(m_i)_{i\in \set}$}(7.25,3.85);
\draw [thick, <-] (3.25,3.85)-- node [midway,above]{$(m_i)_{i\in [n]}$}(4.75,3.85);
\draw [thick] (4.75,3.15)--(5.75,3.15)--(5.75,2.15)--(4.75,2.15)--cycle;
\node at (5.25,2.625){$\F^{\coin}$};
\draw [thick, ->] (5.75,2.35)-- node [midway,above]{$(r_i)_{i\in [n]}$}(7.25,2.35);
\draw [thick, <-] (3.25,2.35)-- node [midway,above]{$(r_i)_{i\in [n]}$}(4.75,2.35);

\node [anchor=west] at (0,2.8){$\forall i\in[n]:$};
\node [anchor=west] at (0,2.3){$\quad s_i:=r_i\oplus m_i$};
\node [anchor=west] at (7.5,2.8){$\forall i\in\set:$};
\node [anchor=west] at (7.5,2.3){$\quad s_i:=r_i\oplus m_i$};

\node [anchor=west] at (0,1.8) {output $(s_i)_{i\in[n]}$};
\node [anchor=west] at (7.5,1.8) {output $(s_i)_{i\in\set}$};
\end{tikzpicture}
}
% 	\framebox{\begin{minipage}{0.95\linewidth}
% 			Input: The sender \send has no input and the  receiver \rec has input $\set\subseteq [n]$ of size $k$.
% 			\begin{enumerate}
% 				\item \label{step:uniform_step1} The parties invoke $\O^{e}_{k,n}$ where $\rec$ takes the role of the receiver with inputs $\set$. \send receives $(m_i)_{i\in [n]}$ and \rec receives $(m_i)_{i\in \set}$ where $m_i\in\{0,1\}^\ell$.
% 				\item \label{step:uniform_step3} The parties invoke $\O^{\coin}$ to receive $n\ell$ coins $r_1,...,r_n\in \{0,1\}^\ell$. 
% 				\item \send outputs $(s_i)_{i\in [n]}$ and \rec outputs $(s_i)_{i\in \set}$ where $s_i:=m_i\oplus r_i$.
% 			\end{enumerate}
% 	\end{minipage}}
	\caption{ Uniform  OT Protocol $\Pi^{\U}_{k,n}$ in the $\OOT^\E,\F^{\coin}$ hybrid (\definitionref{def:ot}).  For all $i\in[n]$, $r_i$, $m_i$ and $s_i$ are in $\bits^\ell$.}
	\label{fig:uniformOT}
\end{figure}


\begin{lemma}\label{lem:EtoU}
	$\Pi^{\U}$ of \figureref{fig:uniformOT} realizes the Uniform Ideal OT $\OOT^{\U}$ (\definitionref{def:ot}) with unconditional security in the $\OOT^\E,\F^{\coin}$ hybrid.
\end{lemma}
\iffullversion
\begin{proof}
	First let us consider any corrupt sender $\Adv_1$. We construct a new adversary $\Adv_1'$ which interacts with functionality $\OOT^{\U}$ and produces an identical output distribution as $\Adv_1$.  $\Adv_1'$ plays the role of $\OOT^\E,\F^{\coin}$ and $\rec$ in $\Pi^{\U}_{k,n}$ while running $\Adv_1$. $\Adv_1'$ receives the endemic OT strings $m_1,...,m_n$ from $\Adv_1$% in \stepref{step:uniform_step1}
.
	
	
	$\Adv_1'$ invokes $\OOT^{\U}$ as the sender and receives $s_1,...,s_n$ in response. When $\Adv_1$ invokes $\F^{\coin}$% during \stepref{step:uniform_step3}
, $\Adv_1'$ sends $r_1,...,r_n$ to $\Adv_1$ on behalf of $\F^{\coin}$ where $r_i:=m_1\oplus s_i$ and outputs whatever $\Adv_1$ outputs. Observe that $s_i$ is sampled uniformly at random by $\OOT^{\U}$ and is independent of the transcript of $\Adv_1$ (conditioned on $r_i$). Therefore computing $r_i:=m_1\oplus s_i$ induces an identical distribution as $r_i\gets\{0,1\}^\ell$.
	
	Now let us consider a corrupt receiver $\Adv_2$. We construct a new adversary $\Adv_2'$ which interacts with $\OOT^{\rec}$ and produces an identical output as $\Adv_2$. $\Adv_2'$ plays the role of $\OOT^\E,\F^{\coin}$ and $\send$ in $\Pi^{\U}_{k,n}$ while running $\Adv_2$. $\Adv_2'$ receives the set $\set\subset[n]$ of size $k$ and the endemic OT strings $(m_i)_{i\in \set}$ from $\Adv_2$% in \stepref{step:uniform_step1}
. 
	
	
	$\Adv_2'$ invokes  $\OOT^{\U}$ as the receiver with input $\set$ and receives $(s_i)_{i\in\set}$ in response. When $\Adv_2$ invokes $\F^{\coin}$% during \stepref{step:uniform_step3} 
	$\Adv_2'$ sends $r_1,...,r_n$ to $\Adv_1$ on behalf of $\F^{\coin}$ where $r_i:=m_1\oplus s_i$ for $i\in \set$ and otherwise sets $r_i$ as the output of $\F^\coin$. $\Adv_2'$ outputs whatever $\Adv_2$ outputs.
	
	The transcripts of these two interactions are identical except for the messages $(r_i)_{i\in \set}$. In the real interaction $r_i$ is sampled uniformly at random by $\F^{\coin}$ as opposed to $\Adv_2'$ computing $r_i:=m_i\oplus s_i$. Observe that $s_i$ is sampled uniformly at random by $\OOT^{\U}$ and is independent of the transcript of $\Adv_2$ (conditioned on $r_i$). Therefore computing $r_i:=m_1\oplus s_i$ induces an identical distribution as $r_i\gets\{0,1\}^\ell$.

	Therefore, any distinguishing advantage adversary $\Adv_1$ or $\Adv_2$ produces in protocol $\Pi^{\U}$ implies that $\Adv_1'$ or $\Adv_2'$ would produce the same advantage against the instantiation of $\OOT^\E$ or $\F^\coin$, i.e. $\negl$.
	
\end{proof}
\else
\begin{proof}[sketch]
	Consider a corrupt \send. The simulator calls $\OOT^{\U}$ and receives $s_1,...,s_n$. When \send invokes $\F^{\coin}$, $\Adv_1'$ sends $r_i:=m_1\oplus s_i$ to $\Adv_1$ on behalf of $\F^{\coin}$. Due to $s_i$ is sampled uniformly this change results in the same distribution. 
	For a corrupt receiver, the simulator effectively does the same. First receive $\set$ and send it to $\OOT^{\U}$ and receive $(s_i)_{i\in\set}$. Then have  $\F^{\coin}$ output $r_i:=m_1\oplus s_i$ for $i\in\set$ and uniform otherwise.
\pe
\end{proof}
\fi























\begin{figure}[t]
\centering
\framebox{
\begin{tikzpicture}
\node [anchor=west] at (0,5) {Sender:};
\node [anchor=west] at (7.5,5) {Receiver:};
\draw [thick] (4.75,4.65)--(5.75,4.65)--(5.75,3.65)--(4.75,3.65)--cycle;
\node at (5.25,4.125){$\OOT^\U$};
\draw [thick, <-] (5.75,4.5)-- node [midway,above]{$[k]$}(7.25,4.5);
\draw [thick, ->] (5.75,3.85)-- node [midway,above]{$(m_i)_{i\in [k]}$}(7.25,3.85);
\draw [thick, <-] (3.25,3.85)-- node [midway,above]{$(m_i)_{i\in [n]}$}(4.75,3.85);
\node [anchor=west] at (0,3.5) {output $(m_i)_{i\in[k]}$};
\node [anchor=west] at (7.5,3.5) {output $(m_i)_{i\in[k]}$};
\end{tikzpicture}
}
% 	\framebox{\begin{minipage}{0.95\linewidth}
% 			Input: The sender \send and receiver \rec have no input.
% 			\begin{enumerate}
% 				\item \label{step:coin_step1} \rec sends $\set:=[k]$ to $\O^{u}_{k,n}$ and receives $(s_i)_{i\in \set}$ in response. \send receives $(s_i)_{i\in [n]}$. 
% 				\item \label{step:coin_step2} Both parties output the bits of $(s_i)_{i\in [k]}$.
% 			\end{enumerate}
% 	\end{minipage}}
	\caption{Coin flipping Protocol $\Pi^{\coin}$ in the $\OOT^{\U}$ hybrid. For all $i\in[n]$, $m_i$ is in $\bits^\ell$.}
	\label{fig:coinFlip}
\end{figure}


\begin{lemma}\label{lem:OTtoC}
	$\Pi^{\textsf{coin}}$ of \figureref{fig:coinFlip} realizes an ideal coin flipping protocol with unconditional security  in the $\O^{\U}$ hybrid  (\definitionref{def:ot}).
\end{lemma}
\begin{proof}
	Follows straigthforwardly from the definition of $\OOT^{\U}$ and the ideal coin tossing functionality $\F^{\coin}$, which outputs a random string to both parties.
	\pe
\end{proof}







% \begin{figure}[t]
% \centering
% \framebox{
% \begin{tikzpicture}
% \node [anchor=west] at (0,5.5) {Sender:};
% \node [anchor=west] at (7.5,5.5) {Receiver:};
% \draw [thick] (4.75,4.65)--(5.75,4.65)--(5.75,3.65)--(4.75,3.65)--cycle;
% \node at (5.25,4.125){$\OOT^{\rec}$};
% \draw [thick, <-] (5.75,4.5)-- node [midway,above]{$\set,(s_i)_{i\in\set}$}(7.25,4.5);
% \draw [thick, ->] (5.75,3.85)-- node [midway,above]{$(m_i)_{i\in \set'}$}(7.25,3.85);
% \node [anchor=west] at (0,3.5) {$\forall i\in[n]$};
% \node [anchor=west] at (0,3) {$\quad s_i:=H(s,m_i)$};
% \draw [thick, ->] (3.25,5)-- node [midway, above]{$c:=\com(s)$} (7.25,5);
% \draw [thick, ->] (3.25,3)-- node [midway, above]{$\open(c)$} (7.25,3);
% \node [anchor=west] at (0,5) {$s\leftarrow\bits^\sec$};
% \draw [thick, <-] (3.25,3.85)-- node [midway,above]{$(m_i)_{i\in [n]}$}(4.75,3.85);
% \node [anchor=west] at (0,2) {output $(s_i)_{i\in[n]}$};
% \node [anchor=west] at (7.5,2) {output $(s_i)_{i\in\set}$};
% \node [anchor=west] at (7.5,3) {$\forall i\in\set$};
% \node [anchor=west] at (7.5,2.5) {$\quad s_i:=H(s,m_i)$};
% \end{tikzpicture}
% }
% % 	\framebox{\begin{minipage}{0.95\linewidth}
% % 			Input: The sender \send and receiver \rec have no input.
% % 			\begin{enumerate}
% % 				\item \label{step:uniformSelect_step1} \rec uniformly samples $\set'\gets \mathbb{P}([n])$ s.t. $|\set'|=k$. \rec sends $\set'$ to $\OOT^{\E}$ and receives $(s_i')_{i\in \set'}$ in response. \send receives $(s_i')_{i\in [n]}$. 
% % 				\item \label{step:uniformSelect_step2} \send samples a random permutation $\pi : [n] \rightarrow [n]$ and sends it to \rec.
% % 				\item \label{step:uniformSelect_step3} \send outputs $(s_i)_{i\in [n]}$ and \rec outputs $(\set, (s_i)_{i\in\set})$ where $s_i:=s_{\pi(i)}'$ and $\set:=\{j \mid \exists i, j=\pi(i)\}$.
% % 			\end{enumerate}
% % 	\end{minipage}}
% 	\caption{Uniform $k$-out-of-$n$ OT protocol $\Pi^{\U}_{k,n}$ in the $\OOT^{\rec}$ hybrid and random oracle model.}
% 
% \label{fig:RtoU}
% \end{figure}

