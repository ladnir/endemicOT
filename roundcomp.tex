\section{Lower Bound on the Round Complexity of Sender and Receiver Chosen Message Security}\label{sec:roundcomp}


In \lemmaref{lem:nosendtweak}, we state that there cannot be an two message OT that achieves sender chosen message security where the sender sends its message first. Here we give the proof. 

\begin{proof}
We show that the most general notion of OT, one out of two OT is impossible.  
For a two message OT where the sender sends its message first, the sender's message $\mes_{\send}$ is a function $f_{\send}$ on input $\tape_{\send}$ and some auxiliary input $\aux$. A sender could sample $s_0,s_1$ during the protocol or receive them as input. We use $\aux$ to cover the second case. Further, there is a function $f_{\rec}$ that takes the random tape $\tape_{\rec}$, $\mes_{\send}$ and choice bit $b$ of $\rec$. Finally, there are two functions $f_{\OT,\send}(\tape_{\send},\mes_{\rec},\aux)$ that outputs $(s_0,s_1)$ and $f_{\OT,\rec}(\tape_{\rec},\mes_{\send},b)$ that outputs $s_b$. 

First, we assume that \send is committed to $(s_0,s_1)$ given $\mes_{\send}$, i.e. there is a $(s_0,s_1)$ such that
$$
\Pr_{\mes_{\rec}, (\tape_{\send},\aux)}[f_{\OT,\send}(\tape_{\send},\mes_{\rec},\aux)=(s_0,s_1)\mid \mes_{\send}=f_{\send}(\tape_{\send},\aux)]\geq\frac{3}{4}.
$$

In this case, a malicious receiver can break the security as follows. It selects two random tapes $\tape_{\rec,1}$, $\tape_{\rec,2}$, two choice bits $b_1=0$, $b_2=1$ and computes for all $i\in[2]$, $\mes_{\rec,i}= f_{\rec}(\tape_{\rec,i},\mes_{\send},b_i)$ and $s_{b_{i},i}=f_{\OT,\rec}(\tape_{\rec,i},\mes_{\send},b_i)$. It outputs $(s_{0,1},s_{1,2})$ as a guess for $s_0,s_1$.

Let the scheme be $\delta$ correct, for $\delta\geq 1-\negl$. Then, the probability that the first malicious receiver reconstructs $(s_0,s_1)$ correctly is lower bounded using Jensen's inequality by
$$
\Pr[(s_{0,1},s_{1,2})=(s_0,s_1)]\geq \left(\frac{3}{4}\delta\right)^2> \frac{1}{2},
$$
where $(s_0,s_1)=f_{\OT,\send}(\tape_{\send},\mes_{\rec},\aux)$. A malicious receiver interacting with the ideal OT can achieve this at most with probability $\frac{1}{2}$. Hence, there is a distinguisher that breaks the sender chosen message security of the OT.

Now assume that for any $(s_0,s_1)$,
\begin{eqnarray}\label{eqn:noncomm}
\Pr_{\mes_{\rec}, \tape_{\send}}[f_{\OT,\send}(\tape_{\send},\mes_{\rec},\aux)=(s_0,s_1)\mid \mes_{\send}=f_{\send}(\tape_{\send},\aux)]<\frac{3}{4}.
\end{eqnarray}
In this case, we show that a malicious receiver can tweak the distribution of $s_0,s_1$.
The malicious receiver uses a hardwired pseudorandom function (PRF) key $\key$ for a PRF $\PRF$ that outputs a single bit.\footnote{OT implies one-way functions and hence also PRFs.} The malicious receiver samples two random tapes $\tape_{\rec,1}$, $\tape_{\rec,2}$, two uniform choice bits $b_1$, $b_2$, computes for all $i\in[2]$ $\mes_{\rec,i}=f_{\rec}(\tape_{\rec,i},\mes_{\send},b_i)$ and $s_{b_i,i}=f_{\OT,\rec}(\tape_{\rec,i},\mes_{\send},b_i)$. If $\PRF_{\key}(s_{b_1,1})=0$ it sends $\mes_{\rec,1}$ to \send and outputs $s_{b_1,1}$ otherwise it sends $\mes_{\rec, 2}$ to \send and outputs $s_{b_2,2}$

 We first give a bound on the probability that $\PRF_{\key}(s_{b_i,i})=0$. Let $\epsilon_{\PRF}$ be a probability such that
$$
\Pr[\PRF_{\key}(s_{b_i,i})=0]= \frac{1}{2}-\epsilon_{\PRF}.
$$
Then there is a distinguisher $\D$ that simply outputs $1$ if $\PRF_{\key}(s_{b_i,i})=0$. Hence,
$$
|\Pr[\D(1^\sec,s_{b_i,i},\PRF_{\key}(s_{b_i,i}))=1]-\Pr[\D(1^\sec,s_{b_i,i},u)=1]|=\epsilon_{\PRF},
$$  
where $u\leftarrow\bits$. Since $\D$ breaks \PRF with probability $\epsilon_{\PRF}$ and \PRF is secure, $\epsilon_{\PRF}$ is negligible. By $(\ref{eqn:noncomm})$, it holds with at least probability $\frac{3}{16}$ that $s_{b_1,1}\neq s_{b_2,2}$. Therefore, the probability that for the output $s_{b_i,i}$ of the malicious holds $\PRF_{\key}(s_{b_i,i})=0$ is at least
\begin{eqnarray*}
\lefteqn{\Pr[\PRF_{\key}(s_{b_i,i})=0]}\\
&=&\Pr[\PRF_{\key}(s_{b_1,1})=0]+\Pr[\PRF_{\key}(s_{b_1,1})=1\wedge\PRF_{\key}(s_{b_2,2})=0]\\
&\geq&  \left(\frac{1}{2}-\epsilon_{\PRF}\right)+\left(\frac{1}{2}+\epsilon_{\PRF}\right)\cdot\frac{3}{16}\left(\frac{1}{2}-\epsilon_{\PRF}\right)\\
&=& \frac{1}{2}+\frac{1}{64}-\negl.
\end{eqnarray*}
Since the OT is $\delta=1-\negl$ correct, we get by using a union bound that the malicious receivers output $s_{b_i,i}$ is correct and $\PRF_{\key}(s_{b_i,i})=0$ holds at least with probability $\frac{1}{2}+\frac{1}{64}-\negl$. Hence, $\PRF_{\key}(s_{b_i,i})=0$ holds with a noticeable bias - given $\key$- when the malicious receiver interacts with an honest sender. If there is a distribution of $(s_0,s_1)$ from which the ideal OT samples with the same bias, then there is a distinguisher $\D$ that breaks the security of \PRF. $\D$ samples from this distribution, queries $\PRF$ on the samples and outputs $1$ if the query returns $0$. 

Since there is no such a distribution for a secure PRF, there is no adversary $\Adv'$ that creates the same output distribution when interacting with the ideal OT as when the malicious receiver interacts with the honest sender. Hence, the OT is not sender chosen message secure. 
\pe
\end{proof}

The following proof is the proof for \lemmaref{lem:norectweak}, which states that there cannot be an two message OT that achieves receiver chosen message security where the receiver sends its message first.

\begin{proof}
Again, we rule out the most general notion of OT, one out of two OT. We follow a similar strategy as in the previous lemma.
A two message OT where the receiver sends its message first has the following structure. The receiver's message $\mes_{\rec}$ is a function $f_{\rec}$ on input $\tape_{\rec}$, $b$ and some auxiliary input $\aux$. Further, there is a function $f_{\send}$ that takes the random tape $\tape_{\send}$ and $\mes_{\rec}$ as input. Finally, there are two functions $f_{\OT,\send}(\tape_{\send},\mes_{\rec})$ that outputs $(s_0,s_1)$ and $f_{\OT,\rec}(\tape_{\rec},\mes_{\send},b,\aux)$ that outputs $s_b$. 

We distinguish two cases. In case one,  we assume that \rec is committed to $s_b$ given $\mes_{\rec}$, i.e. there is a $s_b$ such that
$$
\Pr_{\mes_{\send}, (\tape_{\rec},b,\aux)}[f_{\OT,\rec}(\tape_{\rec},\mes_{\send},b,\aux)=s_b\mid \mes_{\rec}=f_{\rec}(\tape_{\rec},b,\aux)]=\frac{3}{4}+\alpha,
$$
for $\alpha\geq 0$.
Further, $s_{\overline b}$ should not be determined by $\mes_{\rec}$. Let $\ell$ be the length of $s_0$, $s_1$, then if there is a $s_{\overline b}$ s.t. 
$$
\Pr_{\tape_{\send}}[s_{\overline b}\text{ is output of }f_{\OT,\send}(\tape_{\send},\mes_{\rec})\text{ for bit }\overline b\mid \mes_{\rec}]=\frac{1}{2^\ell}+\epsilon,
$$
then a malicious receiver can sample $\tape_{\send}$ and compute $f_{\OT,\send}(\tape_{\send},\mes_{\rec})$ to learn $s_{\overline b}$ with probability $\frac{1}{2^\ell}+\epsilon-(1-\delta)$, where OT is $\delta=1-\negl$ correct. Since the ideal primitive samples $s_{\overline b}$ at uniform, the malicious receiver breaks the OT with probability $\epsilon-\negl$. Therefore, $\epsilon=\negl$. 

But now, a malicious sender can sample two random tapes $\tape_{\send,1}$, $\tape_{\send,2}$
and compute for all $i\in[2]$, $(s_{0,i},s_{1,i})=f_{\OT,\send}(\tape_{\send,i},\mes_{\rec})$ and checks for all $i\in\bits$ whether $s_{i,1}=s_{i,2}$. It outputs a random $b'$ if it holds for both or no $i\in\bits$, otherwise it outputs $b'$ such that $s_{b',1}=s_{b',2}$. It holds that
$$
\Pr[s_{b,1}=s_{b,2}]= \left(\frac{3}{4}+\alpha\right)^2+\left(\frac{1}{4}-\alpha\right)^2\frac{1}{2^\ell-1}\\
\geq \frac{9}{16}
$$
%\frac{9}{16}+\frac{1}{16(n-1)}+\frac{3(n-1)+1}{2(n-1)}\alpha+\frac{n}{n-1}\alpha^2
and
$$
\Pr[s_{\overline{b},1}=s_{\overline{b},2}]\geq \frac{1}{2^\ell}-\epsilon.
$$
Therefore,
\begin{eqnarray*}
\lefteqn{\Pr[b=b']}\\
&=&\frac{1}{2}\Pr[\nexists! i:s_{i,1}= s_{i,2}]+\Pr[s_{b,1}=s_{b,2}\wedge s_{\overline b,1}\neq s_{\overline b,2}]-(1-\delta)\\
&\geq& \frac{2^\ell-1}{2^{\ell+1}}-\frac{2^\ell-2}{2^{\ell+1}}\Pr[s_{b,1}=s_{b,2}]+\frac{2^\ell-1}{2^\ell}\Pr[s_{b,1}=s_{b,2}]
-\negl\\
&\geq & \frac{2^{\ell}-1}{2^{\ell+1}}+\frac{2^{\ell}}{2^{\ell+1}}\frac{9}{16}-\negl\geq
\frac{1}{2}+\frac{1}{32}-\frac{1}{2^{\ell+1}}+\frac{1}{4}-\negl
\end{eqnarray*}
where we apply a union bound to argue that the outputs are correct and corresponds to the receivers choice. Given that $\ell\geq 1$, the malicious sender guesses $b$ correctly with at least probability $\frac{1}{2}+\frac{1}{32}-\negl$. Since in the ideal model, an adversary can guess $b$ only with probability $\frac{1}{2}$, this constitutes a break of receiver chosen message security.

In the case two, for any $s_b$,
$$
\Pr_{\mes_{\send}, (\tape_{\rec},b,\aux)}[f_{\OT,\rec}(\tape_{\rec},\mes_{\send},b,\aux)=s_b\mid \mes_{\rec}=f_{\rec}(\tape_{\rec},b,\aux)]<\frac{3}{4}.
$$
Similar as in the proof of Lemma~\ref{lem:nosendtweak}, we argue that a malicious sender can tweak the output distribution of $(s_0,s_1)$. Due to the similarity, we only exhibit a brief version. Again we hardwire a PRF key $\key$ for a PRF with a single output bit. Given $\mes_{\rec}$, the malicious sender samples two random tapes $\tape_{\send, 1}$ and $\tape_{\send,2}$, computes for all $i\in[2]$ $(s_{0,i},s_{0,i})=f_{\OT,\send}(\tape_{\send,i},\mes_{\rec})$. If $\PRF_{\key}(s_{0,1},s_{1,1})=0$ output $(s_{0,1},s_{1,1})$ and send $\mes_{\send,1}=f_{\send}(\tape_{\send,1},\mes_{\rec})$ to \rec. Otherwise output $(s_{0,2},s_{1,2})$ and send $\mes_{\send,2}=f_{\send}(\tape_{\send,2},\mes_{\rec})$ to \rec. This way, $\Pr_{\key}(s_0,s_1)=0$ holds for the malicious sender's output $(s_0,s_1)$ with probability 
\begin{eqnarray*}
\lefteqn{\Pr[\PRF_{\key}(s_0,s_1)=0]}\\
&=&\Pr[\PRF_{\key}(s_{0,1},s_{1,1})=0]+\Pr[\PRF_{\key}(s_{0,1},s_{1,1})=1\wedge\PRF_{\key}(s_{0,2},s_{1,2})=0]\\
&\geq&  \left(\frac{1}{2}-\epsilon_{\PRF}\right)+\left(\frac{1}{2}+\epsilon_{\PRF}\right)\cdot\frac{3}{8}\left(\frac{1}{2}-\epsilon_{\PRF}\right)\\
&=& \frac{1}{2}+\frac{1}{32}-\negl,
\end{eqnarray*}
unless one breaks the security of the PRF. As previously, this constitutes an attack against the receiver chosen message security.
\pe
\end{proof}


