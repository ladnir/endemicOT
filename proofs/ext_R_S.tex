
\begin{proof}\label{proof:ext_R_S}
		\begin{claim}[Malicious Sender Security]\label{claim:ext-R-S-MalSender}
		$\Pi^{\textsf{ext-R}}$ satisfies security against a malicious sender (\definitionref{def:otSec}) with respect to the $\OOT^\send$ functionality.
	\end{claim}
	\begin{proof}
		%The simulation follows the same strategy as \lemmaref{lem:ext-E} except now $\Adv$ is allowed to sample $k$ and influence the messages being output by the parties. In particular, after \Adv sends $k$ to $\Adv'$, $\Adv'$ defines the circuits $\mathcal{M}_i(x)=\pi_k(y)+y$ where $y:=\tt_i + \bb \odot (\cc_i + \mathcal{C}(map(x)))$ and sends it to $\OOT^\send$ and the input of \send.
		
		The simulation follows essentially the same strategy as \lemmaref{lem:ext-E}. Consider the following hybrids which will define the simulator $\Adv'$. 
		\begin{enumerate}[leftmargin=1.8cm]
			\item[Hybrid 1.] $\Adv'$ internally runs \Adv while plays the role of $\rec$ and base OT oracle $\OOT=\OOT^\rec$. For $j\in[\nc]$, $\Adv'$ receives $(\bb'_j,\tt^j_{\bb_j})\in [2]\times \mathbb{F}^{m'}_2$ from \Adv in \stepref{step:extInit} where $\bb_j:=\bb_j'-1$. $\Adv'$ uniformly samples $\tt^j_{1-\bb_j}$ as $\OOT=\OOT^\E$ would. $\Adv'$ sends $(\bb', \{\tt^j_{\bb}\})$ to $\Adv$ on behalf of \OOT. $\Adv'$ outputs whatever \Adv outputs. The view of $\Adv$ is unmodified.
			
			\item[Hybrid 2.] For \stepref{step:extSendU} $\Adv'$ does not sample $\tt^j_{1-\bb_j}$ and instead uniformly samples $U\gets\mathbb{F}^{m'\times \nc}_2$. $\Adv'$ sends $U$ to \Adv and then computes $Q$ as \send would. The view of $\Adv$ is identically distributed. This follows from the fact that $\tt^j_{1-\bb_j}$ is uniformly distributed in the view of \Adv and masks the $j$-th column of $U$ in the previous hybrid. 
			
			\item[Hybrid 3.] For \stepref{step:consistency} $\Adv'$ simulates the consistency proof. This change is indistinguishable. 
			
			\item[Hybrid 4.]\label{hybrid:mmmm} $\Adv'$ receives $k$ from $\Adv$ as specified in \definitionref{def:ext_R_S}. For each row $\qq_i$, $\Adv'$ defines the circuit $\mathcal{M}_i:[N]\rightarrow\{0,1\}^\kappa$ such that on input $j\in[N]$ it outputs $\H( \qq_i+\bb\odot \mathcal{C}(map(j)))$. $\Adv'$ sends $\mathcal{M}_i$ to the ideal oracle $\OOT^\E$ as the sender's input to the $i$-th OT instance. This change allows the ideal oracle to output the same distribution as the real protocol. The view of $\Adv$ is unmodified.
			
			Let $y_{j}= \qq_i+\bb\odot \mathcal{C}(map(j))=\tt_i + \bb \odot (c_i + \mathcal{C}(map(j))$ and note that $\Adv$ can influence $\mathcal{M}_i(j)=\H(y_j)=\pi_k(y_j)+y_j$ by choosing $k,\bb$ and the bits $\{\tt_i[j] \mid \bb_j=0\}$.
			
			\item[Hybrid 5.] $\Adv'$ does not take the input of \rec. \rec only interacts with $\OOT^\E$. This change is identically distributed since $\Adv'$ was not using the input of \rec.
		\end{enumerate}
		\pe
	\end{proof}
	\begin{claim}[Malicious Receiver $\OOT^\U$-Security]\label{claim:ext-R-S-MalReceiver}
		$\Pi^{\textsf{ext-R}}$ satisfies security against a malicious receiver (\definitionref{def:otSec}) with respect to the $\OOT^\U$ functionality.
	\end{claim}
	\begin{proof}
		The simulation also follows a similar strategy as \lemmaref{lem:ext-E}. Consider the following hybrids which will define the simulator $\Adv'$. 
		\begin{enumerate}[leftmargin=1.8cm]
			\item[Hybrid 1.] $\Adv'$ internally runs \Adv while plays the role of $\send$ and base OT oracle $\OOT=\OOT^\rec$. $\Adv'$ uniformly samples $\{\tt^j_0,\tt^j_1\}_{i\in\nc}$ and sends them to \Adv in \stepref{step:extInit}. $\Adv'$ samples $\bb$ as \send would. $\Adv'$ outputs whatever \Adv outputs. The view of $\Adv$ is unmodified.
			
			\item[Hybrid 2.] In \stepref{step:extSendU} $\Adv'$ receives $U$ from $\Adv$.  $\Adv'$ computes $C$ and $Q$ using $\tt_0^j,\tt_1^j,\bb$. $\Adv'$ performs the proof of \stepref{step:consistency} as \send would. If the proof fails, $\Adv'$ aborts as \send would. Otherwise, by the correctness of the proof, $\cc_i$ decodes to $\ww_i$ and computes $x_i$ s.t. $\ww_i=map(x_i)$. 
			
			For all $i\in[m]$, $\Adv'$ defines the circuit $\mathcal{S}_i:[N]\rightarrow\{0,1\}$ which outputs 1 at $x_i$ and 0 otherwise. $\Adv'$ sends $\mathcal{S}_i$ and then $(\textsf{Output}, x_i)$ to $\OOT^\send$ as the receiver's input to the $i$-th  $\OOT^\send$ instance which responds with $v_{i,x_i}$. The view of \Adv is unmodified.
			
			
			\item[Hybrid 3.] $\Adv'$ then uniformly samples $k\gets\{0,1\}^\kappa$ as \send would and defines the ideal permutation $\pi_k$. If $\pi_k$ has been queries by $\Adv$, then $\Adv'$ aborts. The probability of this event is negligible due to $k$ being uniformly sampled from $\{0,1\}^\kappa$. Otherwise, before sending $k$ to \send, $\Adv'$ programs $\pi_k$ s.t.  $\pi_k(\tt_i)=\vv_{i, x_i}+\tt_i$. Conditioned on these input/outputs not colliding for $i\in[m]$, which happens with overwhelming probability, this modification is identically distributed due to $\vv_{i, x_i}\gets\{0,1\}^\kappa$ being sampled uniformly by $\OOT^\U$.
			
			
			\item[Hybrid 4.]\label{hybrid:simOutputR} Assuming $\Adv'$ did not abort in \stepref{step:consistency}, let $E=\{j \mid \exists i\in[m], (\cc_i\oplus \mathcal{C}(\ww_i))_j = 1\}$ index the columns of $C$ where $\Adv$ added an error to any codeword $\cc_i$ (w.r.t $\ww_i$). By the correctness of \stepref{step:consistency}, it holds that $E\subseteq B_0$, otherwise the consistency proof would have failed. By passing the consistency proof, \Adv learns what $\bb_j=0$ for all $j\in E$. Similarly, the probability of passing the check and $\Pr[|E|=d]=\Pr[\bb_j=0 \mid \forall j\in E]=2^{-d}$ due to the proof being independent of $\bb$. We will see that this is equivalent to \Adv simply guessing $E$ (which is correct with the same probability) and then being honest. 
			
			For all $w\neq \ww_i$, \Adv  has $\negl$ probability of computing $g=\qq_i+\bb\odot \mathcal{C}(w)$. If this was not the case, then \Adv  could compute
			\begin{align*}
			g +\tt_i &=\qq_i+\bb\odot \mathcal{C}(w)+\tt_i\\
			&=\cc_i\odot\bb+\tt_i+\bb\odot \mathcal{C}(w)+\tt_i\\
			&=(\cc_i+\mathcal{C}(w))\odot \bb\\
			&=(\mathcal{C}(\ww_i)+\mathcal{C}(w))\odot \bb
			\end{align*}
			This last equality holds due to $\Adv'$ aborting if $(\cc_i+\mathcal{C}(\ww_i))\odot \bb\neq 0$. Recall that $\mathcal{C}$ has minimum distance $\dc\geq \kappa$ and therefore computing $g$ is equivalent $\Adv$ guessing $\dc\geq \kappa$ bits of $\bb$ which happens with probability $2^{-\dc}\leq 2^{-\kappa}$.  As such, the probability that \Adv has made a query of the form $\pi_k(\qq_i+\bb\odot \mathcal{C}(w))$ for $w\neq \ww_i$ is also negligible. If such as query does happen $\Adv'$ aborts. This hybrid is indistinguishably distributed from the previous. 
			
			
			
			\item[Hybrid 5.] \label{hybrid:simOutput2R} When \send makes an $\pi_k$ query of the form $\pi_k(h)$ which they have not previously been queried, $\Adv'$ must determine if there is a unique $w\in\mathbb{F}^\nc_2,i\in[m]$ such that $h=\qq_i+\bb\odot \mathcal{C}(w)$. For the sake of contradiction, let us assume there exists any two $i,i'\in[m]$ or  $w,w'\in\mathbb{F}^\kc_2$ which result in the same input to $\pi_k$. If $i=i'$ and $w=w'$, then a unique $(i,w)$ exist. Otherwise,
			\begin{align*}
			\tt_i + \bb \odot (c_i + \mathcal{C}(w)) &= \tt_{i'} +\bb \odot (c_{i'} + \mathcal{C}(w'))\\
			%\tt_i + \bb \odot \mathcal{C}(w) &=\tt_{i'} 		+	\bb \odot \mathcal{C}(w') \\
			\bb \odot (\mathcal{C}(w) + \mathcal{C}(w') + \cc_i +\cc_{i'})  &= \tt_{i} +\tt_{i'}  \\
			\bb\odot\delta&=\tt_i+\tt_{i'}			
			\end{align*}			
			where $\delta:=\mathcal{C}(w) + \mathcal{C}(w')+\cc_i+\cc_{i'}$. If $i=i'$, then it must hold $\bb \odot (\mathcal{C}(w) + \mathcal{C}(w'))=0$ for $w\neq w'$.  Recall that $\mathcal{C}$ by construction has minimum distance $\dc\geq\kappa$ and that $\bb$ is uniformly distributed. Let $E=\{i \mid \delta_i=1\}$, then $|E|\geq \dc\geq \kappa$ and for the above to hold we require $\bb_i=0 \mid \forall i\in E$ which occurs with probability $\Pr[\bb_i=0 \mid \forall i\in E]=2^{-|E|}\leq 2^{-\dc}\leq2^{-\kappa}$.  Therefore with overwhelming probability a unique $(i,w)$ exist if $i=i'$. 
			
			Otherwise, let  $B_j:=\{i \mid \bb_i=j\}$ and due to  \stepref{step:consistency} it holds that for all $i\in[m]$, $\cc_i\odot\bb$ erasure decodes to $\ww_i$ with $B_0$ indexing the erasures. Therefore, by the linearity of $\mathcal{C}$, $\delta$ erasure decodes to some $w^*$ with $B_0$ indexing the erasures s.t. $\bb \odot c^*=\bb\odot \delta$ where $c^*:= \mathcal{C}(w^*)$. 
			
			
			Fixing some $i,i'$, the probability $\bb\odot c^* = \tt_{i} +\tt_{i'}$  is $p_0=\Pr[(\tt_{i} +\tt_{i'})_{\ell}=0 \mid \forall\ell \in B_0]\leq 2^{-|B_0|}$ times $p_1=\Pr_{c^*}[(\tt_{i} +\tt_{i'}+c^*)_\ell = 0 \mid \forall \ell \in B_1]\leq N2^{-|B_1|}$. Therefore, the probability that $i\neq i'$ and $w\neq w'$ is at most the union bound over all $i,i'\in[m]$,
			\begin{equation}\label{eq:unique}
				\Pr_{i,i',c^*}[\bb\odot c^* \leq \tt_{i} +\tt_{i'}]\leq m^2p_0p_1=m^2N2^{-\nc}
			\end{equation}
			which is negligible\footnote{Note, $N$ is assumed to be polynomial. This is true in the target use case where $N=2$ and $\nc=\kappa$. }.  Therefore we conclude that $(i,w)$ is unique if such a pair exists.
			
			
			If so then $\Adv'$ can use Gaussian elimination to identify it. In particular, $\Adv'$ computes $h+\qq_i$ for all $i\in[m]$ and checks that $(h+\qq_i)_\ell =0$ for all $\ell\in B_1$ and if so tries erasure decodes $h+\qq_i$ to $w$ where the erasures are index by $B_0$. For $h+\qq_i$ this will happen and $\Adv'$ computes $x$ s.t. $map(x)=w$ and sends $(\textsc{Output}, x)$ to the $i$-th instance of $\OOT^\send$ and receives $\vv_{i,x}\gets\{0,1\}^\ell$ in response.  Let $y_{i,x}:=h=\tt_i + \bb (\cc_i + \mathcal{C}(map(x)))$.
			
			$\Adv'$ programs $\pi_k(y_{i,x})=\vv_{i,x}+y_{i,x}$. Programming $\pi_k$ requires the input/output pair $(y_{i,x},\vv_{i,x}+y_{i,x})$ to have not previously been queried on $\pi_k,\pi_k^{-1}$.  It is easy to verify that with overwhelming probability $\pi_k^{-1}(\vv_{i,x}+y_{i,x})$ has not been queried since $\vv_{i,x}$ is uniformly distributed. 
			
			In the other direction, $y_{i,x}$  could have been queried in two ways. 1) $\D$ or \Adv guessed it which is negligible as discussed in \hybridref{hybrid:simOutputR}. 2) $\D$ inverted $\vv_{i',x'}:=\H(y_{i',x'})=\pi_k(y_{i',x'})+y_{i',x'}$ and then recovered $\bb$. However, $v=\pi_k(y)+y$ is preimage resistant\cite{C:BlaRogShr02,SP:Winternitz} which informally follows from the difficulty of finding an input to the random permutation $\pi_k$ which differs from $v$ by itself.
			%However, we also must show that $y_{i,x}:=h=\tt_i + \bb (\cc_i + \mathcal{C}(map(x)))$ has not been queried by $\Adv$ or $D$. The view of $D$ contains $\tt_i$ and polynomial many outputs $\vv_{i,x}=\pi_k(y_{i,x}) + y_{i,x}$. 
			%The distribution of $\pi$ after being programmed is identical since the input or output has not previously been queried. In particular, later a distinguisher may be given access to $\pi_k$ and $\pi_k^{-1}$ and many $\vv_{i,x}$ values. However, it remains hard to find a preimage $h$ of $\vv_{i,x}=\pi_k(h)+h$ due to $h$ masking the permutation output\cite{daviesMeyers}.
			
			\item[Hybrid 6.] $\Adv'$ does not take the input of \send and does not program $\pi$ in \hybridref{hybrid:simOutput2R}. \send only interacts with $\OOT^\send$. This change is identically distributed.
		\end{enumerate} 
		\pe
	\end{proof}


	\begin{claim}[Malicious Receiver $\OOT^\send$-Security]\label{claim:ext-R-S-MalReceiver2}
		$\Pi^{\textsf{ext-R}}$ satisfies Security Against a Malicious Receiver (\definitionref{def:otSec}) with respect to the $\OOT^\send$ functionality.
	\end{claim}
	\begin{proof}
		Follows from \lemmaref{lemma:is_a} and the previous claim.
		\pe
	\end{proof}
	\pe
\end{proof}