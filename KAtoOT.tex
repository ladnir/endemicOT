\section{From Key Agreement to Oblivious Transfer}\label{sec:endemicOT}

In \figureref{fig:KAtoOT}, we present the generic construction from any two round key agreement to a two round OT. More efficient is the construction presented in \figureref{fig:oneroundKAtoOT} that transforms a one round key agreement into a one round OT. In \theoremref{thm:KAtoOT}, we show that both constructions yield an endemically secure OT.



\begin{figure}[h!]
\centering
\framebox{
\begin{tikzpicture}
\node [anchor=west] at (-0.5,5) {Sender:};
\node [anchor=west] at (7,5) {Receiver$(\i\in[n])$:};
\node [anchor=west] at (7,4.5){$\forall j\in[n]\setminus\{\i\}: r_{j}\leftarrow\G$};
\node [anchor=west] at (7,4){$\tape_\A\leftarrow\bits^*$};
\node [anchor=west] at (7,3.5){$\mes_\A\leftarrow \UKA.\A(\tape_\A)$};
\node [anchor=west] at (7,3){$r_{\i}:=\mes_\A\ominus\H((r_j)_{j\in[n]\setminus\{\i\}})$};
\draw [thick, <-] (4.75,3)-- node [midway,above]{$(r_j)_{j\in[n]}$}(6.75,3);
\node [anchor=west] at (-0.5,3){$\forall j\in[n]:$};
\node [anchor=west] at (-0.5,2.5){$\quad\mes_{\A,j}:=r_j\oplus\H((r_\ell)_{\ell\in[n]\setminus\{j\}})$};
\node [anchor=west] at (-0.5,2){$\quad\tape_{\B,j}\leftarrow\bits^*$};
\node [anchor=west] at (-0.5,1.5){$\quad\mes_{\B,j}\leftarrow\UKA.\B(\tape_{\B,j},\mes_{\A,j})$};
\draw [thick, ->] (4.75,1.5)-- node [midway,above]{$(\mes_{\B,j})_{j\in[n]}$}(6.75,1.5);
\node [anchor=west] at (7,0.5) {$\key_{\A,\i}:=\UKA.\EK(\tape_\A,\mes_{\B,\i})$};
\node [anchor=west] at (-0.5,1) {$\forall j\in[n]$};
\node [anchor=west] at (-0.5,0.5) {$\quad\key_{\B,j}:=\UKA.\EK(\tape_{\B,j},\mes_{\A,j})$};
\end{tikzpicture}
}
\caption{The figure depicts a $1$ out of $n$ oblivious transfer using as a building block a \UKA and a random oracle $\H:\G^{n-1}\rightarrow\G$, where \G is a group with group operations $\oplus$, $\ominus$. By the correctness of the \UKA scheme, $\key_{\A,\i}=\key_{\B,\i}$ holds.}
\label{fig:KAtoOT}
\end{figure}

\begin{figure}[h!]
\centering
\framebox{
\begin{tikzpicture}
\node [anchor=west] at (-0.5,5) {Sender:};
\node [anchor=west] at (7,5) {Receiver$(\i\in[n])$:};
\node [anchor=west] at (7,4.5){$\forall j\in[n]\setminus\{\i\}: r_{j}\leftarrow\G$};
\node [anchor=west] at (7,4){$\tape_\A\leftarrow\bits^*$};
\node [anchor=west] at (7,3.5){$\mes_\A\leftarrow \UKA.\A(\tape_\A)$};
\node [anchor=west] at (7,3){$r_{\i}:=\mes_\A\ominus\H((r_j)_{j\in[n]\setminus\{\i\}})$};
\draw [thick, ->] (4.75,3.15)-- node [midway,above]{$\mes_{\B}$}(6.75,3.15);
\draw [thick, <-] (4.75,3)-- node [midway,below]{$(r_j)_{j\in[n]}$}(6.75,3);
\node [anchor=west] at (-0.5,3){$\forall j\in[n]:$};
\node [anchor=west] at (-0.5,2.5){$\quad\mes_{\A,j}:=r_j\oplus\H((r_\ell)_{\ell\in[n]\setminus\{j\}})$};
\node [anchor=west] at (-0.5,4.5){$\tape_{\B}\leftarrow\bits^*$};
\node [anchor=west] at (-0.5,4){$\mes_{\B}\leftarrow\UKA.\B(\tape_{\B})$};
\node [anchor=west] at (7,2) {$\key_{\A,\i}:=\UKA.\EK(\tape_\A,\mes_{\B})$};
\node [anchor=west] at (-0.5,2) {$\quad\key_{\B,j}:=\UKA.\EK(\tape_{\B},\mes_{\A,j})$};
\end{tikzpicture}
}
\caption{The figure depicts a $1$ out of $n$ oblivious transfer using as a building block a one-round \UKA and a random oracle $\H:\G^{n-1}\rightarrow\G$, where \G is a group with group operations $\oplus$, $\ominus$. By the correctness of the \UKA scheme, $\key_{\A,\i}=\key_{\B,\i}$ holds.}
\label{fig:oneroundKAtoOT}
\end{figure}

 
\begin{theorem}\label{thm:KAtoOT}
Given a correct and secure [one-round] \UKA scheme, then the $1$ out of $n$ oblivious transfer in [\figureref{fig:oneroundKAtoOT}] \figureref{fig:KAtoOT}   is an Endemic $\OT_{1,n}$ in the programmable random oracle model. 
\end{theorem}

\begin{proof}
The security of the protocols in \figureref{fig:oneroundKAtoOT} and \figureref{fig:KAtoOT} are basically identical except one small difference. Therefore, we focus in this proof on proving the security of the protocol in \figureref{fig:KAtoOT}. We will point out the one small change that one needs to make in order to prove security of the protocol in \figureref{fig:oneroundKAtoOT}. These change occurs when using the key indistinguishability of the \UKA scheme. 
We start with proving that the scheme is secure against a malicious sender. 
\begin{claim}\label{claim:malsender}
Given a $\delta$ correct and $\epsilon$ $(n-1)$-multi instance uniform \UKA scheme, then it holds that in the programmable random oracle model for any ppt adversary \Adv, there exists a ppt adversary \Adv' such that for any ppt distinguisher \D
$$
|\Pr[\D((\Adv,\rec)_{\Pi})=1] -\Pr[\D( (\Adv', \OOT^\E))=1]|\leq \epsilon+(1-\delta),
$$
where all algorithms receive input $1^\sec$ and \rec additionally receives input \set.
\end{claim}

\begin{proof}
We define $\Adv'$ as follows. 
It generates $(r_j)_{j\in[n]}$ by sampling $r_1,\dots, r_n\leftarrow \G$. Then, it samples for all $j\in[n]$, $\tape_{\A,j}\leftarrow\bits^*$ and $\mes_{\A,j}\leftarrow\UKA.\A(\tape_{\A,j})$. Finally it programs the random oracle for all of the $j\in[n]$ points $(r_i)_{i\in[n]\setminus \{j\}}$ such that $r_i\oplus\H((r_i)_{i\in[n]\setminus \{j\}})=\mes_{\A,i}$. Now $\Adv'$ invokes \Adv, answers his random oracle queries straightforwardly, sends $(r_j)_{j\in[n]}$ and receives $(\mes_{\B,j})_{j\in[n]}$ from \Adv. It computes $s_{\A,j}\leftarrow\UKA.\EK(\tape_{\A,j},\mes_{\B,j})$ for all $j\in[n]$ and submits $(s_{\A,j})_{j\in [n]}$ to $\OOT^{\E}$. $\Adv'$ outputs the output of \Adv.

We show, that if there is a distinguisher \D that distinguishes the distribution $(\Adv,\rec)_{\Pi}$ from $(\Adv',\OOT^{\E})$, then there is an distinguisher $\D_{\UKA}$ against the $n$-multi instance uniformity of the \UKA scheme. 

$\D_{\UKA}$ has access to an oracle $\O$ that either outputs uniform strings or messages of the \UKA protocol. For all $j\in[n]\setminus\{i\}$, $\D_{\UKA}$ follows the description of $\Adv'$ with the difference that instead of sampling  $\mes_{\A,j}\leftarrow\UKA.\A(\tape_{\A,j})$, it samples $\mes_{\A,j}$ from $\O$. Given $(r_j)_{j\in[n]\setminus\{i\}}$, it samples $\mes_{\A,i}\leftarrow\UKA.\A(\tape_{\A,i})$ and sets $r_i$ such that $r_i\oplus\H((r_j)_{j\in[n]\setminus \{i\}})=\mes_{\A,i}$. As $\Adv'$, it computes $s_{\A,i}\leftarrow\UKA.\EK(\tape_{\A,i},\mes_{\B,i})$ which is \rec's output. It now invokes distinguisher \D on \rec's output $s_{\A,i}$ and the output of $\Adv$. In the end, it outputs the output of \D.


We now analyze the distributions. 
First, notice that the distribution of $(r_i,\mes_{A,i})$ when sampling $r_i\leftarrow\G$ and then programming the random oracle $r_i\oplus\H((r_i)_{i\in[n]\setminus \{j\}})=\mes_{\A,i}$ is identical to the distribution when sampling $\H((r_i)_{i\in[n]\setminus \{j\}})$ and choosing $r_i$  such that $r_i\oplus\H((r_i)_{i\in[n]\setminus \{j\}})=\mes_{\A,i}$, both are the uniform distribution over $\G\times\G$ conditioned to their sum being $\mes_{\A,i}$. Therefore it follows straightforwardly from the definition of \O, \rec and $\Adv'$ that when \O outputs uniform messages, the output of \Adv is distributed as when interacting with $\rec$ while when \O outputs \UKA messages, it is distributed as the output of $\Adv'$. Hence, if there is a distinguisher $\D$ that distinguishes the output distribution of \Adv given $s_{\A,i}$, i.e.
$$
\epsilon_{\D}\leq |\Pr[\D((\Adv,s_{\A,i})_{\D_{\UKA}^{\O_\A}})=1]-\Pr[\D((\Adv,s_{\A,i})_{\D_{\UKA}^{\O_u}})=1]|
$$
then it implicitly breaks the $(n-1)$-multi instance uniformity of the \UKA protocol, i.e. 
\begin{eqnarray*}
\epsilon&:=& |\Pr[\D_{\UKA}^{\O_{\A}}(1^\sec))=1] -\Pr[ \D_{\UKA}^{\O_u}(1^\sec)=1]|\\
&=&|\Pr[\D((\Adv,s_{\A,i})_{\D_{\UKA}^{\O_{\A}}})=1] -\Pr[ \D((\Adv,s_{\A,i})_{\D_{\UKA}^{\O_{u}}})=1]|\\
&\geq & \epsilon_{\D}.
\end{eqnarray*}

For finishing the proof of the claim, we now need to show that given the output of \Adv, $\D$ cannot distinguish the output of \rec, i.e. $s_{\A,i}$,  from the output of the ideal OT. The ideal OT will output the string that is consistent with \send's view, i.e. $s_{\B,i}$.
Given that \UKA is correct with probability $\delta$, i.e. 
$$
\Pr[s_{\A,i}=\UKA.\EK(\tape_{\A,i},\mes_{\B,i})=\UKA.\EK(\tape_{\B,i},\mes_{\A,i})=s_{\B,i}]= \delta, 
$$
then
\begin{eqnarray*}
\lefteqn{\Pr[ \D((\Adv,s_{\A,i})_{\D_{\UKA}^{\O_{u}}})=1]}\\
&=&\delta\Pr[ \D((\Adv,s_{\B,i})_{\D_{\UKA}^{\O_{u}}})=1]+(1-\delta)\Pr[ \D((\Adv,s_{\A,i})_{\D_{\UKA}^{\O_{u}}})=1\mid s_{\A,i}\neq s_{\B,i}].
\end{eqnarray*}

Hence, given a distinguisher $\D$ with
$$
|\Pr[\D((\Adv,\rec)_{\Pi})=1] -\Pr[ \D((\Adv',\OOT^{\E}))=1]|=\epsilon_{\OT},
$$
and assuming w.l.o.g. 
$$
\Pr[\D((\Adv,\rec)_{\Pi})=1]]>\Pr[  \D((\Adv',\OOT^\E))=1],
$$
 we can lower bound $\epsilon$ of our distinguisher against the \UKA protocol by
\begin{eqnarray*}
\epsilon &=&|\Pr[\D((\Adv,s_{\A,i})_{\D_{\UKA}^{\O_{\A}}})=1] -\Pr[ \D((\Adv,s_{\A,i})_{\D_{\UKA}^{\O_{u}}})=1]|\\
&=& \epsilon_{\OT} +(1-\delta)(\Pr[ \D((\Adv,s_{\B,i})_{\D_{\UKA}^{\O_{u}}})=1]-\Pr[ \D((\Adv,s_{\A,i})_{\D_{\UKA}^{\O_{u}}})=1\mid s_{\A,i}\neq s_{\B,i}])\\
&\geq &\epsilon_{\OT}-(1-\delta).
\end{eqnarray*}
\pe
\end{proof}

We finish the proof of the theorem by showing that the \OT protocol is secure against a malicious receiver.
\begin{claim}\label{claim:malreceiver}
Given a $\delta$ correct, $\epsilon_u$ $Q$ multi-instance uniform, $\epsilon_k$ $(Q,n-1)$ multi-instance key indistinguishable  \UKA scheme, where $Q$ upper bounds the amount of random oracle queries by an adversary then it holds that in the programmable random oracle model for any ppt adversary \Adv, there exists a ppt adversary \Adv' such that for any ppt distinguisher \D
$$
|\Pr[\D((\send, \Adv)_{\Pi})=1] -\Pr[\D((\OOT^\E,\Adv'))=1]|\leq Q(\epsilon_{u}+\epsilon_{k})+(1-\delta),
$$
where all algorithms receive input $1^\sec$.
\end{claim}


\begin{proof}
Again, we start by giving a description of \Adv'. \Adv' guesses a query index $\inda\in[Q]$, where $Q$ is an upper bound on the amount of oracle queries of \Adv. Then \Adv'  invokes \Adv.
When \Adv makes an oracle query $q$ to \H, \Adv' and the query number is less or equal to $\inda$, \Adv responds with a random group element $\H(q)\leftarrow\G$. 
If the query number equals $\inda$, \Adv' stores $(g^*_i)_{i\in[n-1]}:=q_{\inda}$. 
For all following random oracle queries, i.e. the query number $j$ is higher than $\inda$, \Adv ' responds with a random group element $\H(q)\leftarrow\G$ if $q_j$ does not contain $n-2$ elements of $(g^*_i)_{i\in[n-1]}$  and the order is of the form $g^*_1,\dots, g^*_{\ell-1},g,g^*_{\ell},\dots, g^*_{n-1}$, where $g$ is not contained in $(g^*_i)_{i\in[n-1]}$ and $\ell\in[n-1]$. Otherwise \Adv' picks the element $g^*_{\indb_j}\in(g^*_i)_{i\in[n-1]}$ that is missing in $q_j$,  samples random tape $\tape_j\leftarrow\bits^*$ and computes $\mes_{j}\leftarrow\UKA.\A(\tape_j)$. 
It responds with $\H(q_j):=\mes_j\ominus g^*_{\beta_j}$. When \Adv sends $(r_i)_{i\in[n]}$, \Adv' aborts if $q_{\inda}$ is not the first query with $q_{\inda}=(r_i)_{i\in[n]\setminus{j}}$ for a $j\in[n]$. \Adv' sends $i^*$ to the ideal OT $\OOT^{\rec}$. \Adv' computes for all $i\in[n]$ $\mes_{\A,i}:=r_i\oplus\H((r_\ell)_{\ell\in[n]\setminus\{i\}})$, $\tape_{\B,i}\leftarrow\bits^*$ and $\mes_{\B,i}\leftarrow\UKA.\B(\tape_{\B,i},\mes_{\A,i})$. It also computes $s_{\B,i^*}:=\UKA.\EK(\tape_{\B,i^*},\mes_{\A,i^*})$.
\Adv' sends $s_{\B,i^*}$ to $\OOT^{\rec}$, $(\mes_{\B,i})_{i\in[n]}$ to \Adv and outputs the output of \Adv.

Let there be a distinguisher $\D$ with
$$
\epsilon_{\D}:=|\Pr[\D((\send, \Adv)_{\Pi})=1] -\Pr[\D(((s_{\B,i})_{i\in[n]},\Adv')_{\Pi^{ \OOT^\rec , \Adv'}})=1]|,
$$
where $(s_{\B,i})_{i\in[n]}$ are the outputs of $\UKA.\EK(\tape_{\B,i},\mes_{\A,i})$. Then there is a distinguisher $\D_u$ breaking the $Q$-multi instance uniformity of the \UKA protocol. $\D_u$ gets access to an oracle $\O$ which is either outputs uniform messages, i.e. $\O_u$ or messages of  the form $\mes_\A\leftarrow\A(\tape_\A)$ for $\tape_\A\leftarrow\bits^*$. $\D_u$ invokes \D and creates its input as follows. It invokes $\Adv$ and interacts with him as \Adv' does with the difference that $\mes_j$ are requested from $\O$ rather than computing them.  After receiving the output, $\D_u$ uses it as input for $\D$ together with $(s_{\B,i})_{i\in[n]}$, where $s_{\B,i}\leftarrow\UKA.\EK(\tape_{\B,i},\mes_{\A,i})$. $\D_u$ outputs the output of \D. If $\Adv'$ aborts, we let $\D_u$ output $0$.

W.l.o.g. we assume that \Adv queries for all $i\in[i]$ $(r_j)_{j\in[n]\setminus{i}}$ to $\H$. The probability that $D_u$, i.e. \Adv', does not abort is the probability that the guess $\inda$ is correct, i.e.
$$
\Pr[\D_u\text{ does not abort}]=\frac{1}{Q}.
$$

Assuming that $\D_u$, i.e. $\Adv'$ does not abort. Then if $\O$ is oracle $\O_{u}$, then because all $\mes_j$ are uniform, all random oracle queries $q$ are answered with a uniformly random $\H(q)\in\G$.  Otherwise, \Adv' is identical with \send as well as $(s_{\B,i})_{i\in[n]}$ are identical with the output of \send.  
Hence 
\begin{eqnarray*}
\epsilon_{u}&=&|\Pr[\D_u^{\O_{\A}}(1^{\sec})]=1]-Pr[\D_u^{\O_{u}}(1^{\sec})=1]|\\
&= &\frac{1}{Q}|\Pr[\D((s_{\B,i})_{i\in[n]},\Adv)_{\D_u^{\O_{\A}}})=1]-Pr[\D((s_{\B,i})_{i\in[n]},\Adv)_{\D_u^{\O_{u}}})=1]|\\
&\geq&\frac{1}{Q}\epsilon_{\D}.
\end{eqnarray*}



Next, we assume that there is a distinguisher $\D$ with
$$
\epsilon_{\D}:=|\Pr[\D(((s_{\B,i})_{i\in[n]},\Adv)_{\langle \OOT^\rec , \Adv' \rangle})=1] -\Pr[\D((\set_{\B,i^*}, \Adv)_{\langle \OOT^\rec , \Adv' \rangle})=1]|,
$$
where for all $i\in[n]$, $s_i$ is sampled uniformly from the key space of \UKA and $\set_{\B,i^*}:=(s_{1},\dots, s_{i^*-1},s_{\B,i^*},s_{i^*+1},\dots, s_{n})$. Then there is a distinguisher $\D_k$ that breaks the $(Q,n-1)$-multi instance key indistinguishability of the \UKA protocol. $\D_k$ has access to oracles $\O_{\langle \A,\B\rangle}$ and \O which is either $\O_u$ or $\O_k$. In case of a one-round protocol, i.e. \figureref{fig:oneroundKAtoOT}, $\D_k$ has access to $\D_{\A}$ and $\mes_{\B}$ instead of $\O_{\langle \A,\B\rangle}$. In this case, we define $\mes_{\B,j}:=\mes_{\B}$ for all $j\in[Q]$. $\D_k$ invokes $\D$ and creates its input as follows. $\D_k$ invokes \Adv and interacts with it as $\Adv'$ does with the difference, that $\D_k$ generates $\mes_j$ by querying a transcript $\langle \A, \B\rangle=(\mes_{\A,j}',\mes_{\B,j}')$ from $\O_{\langle \A,\B\rangle}$ and setting $\mes_j=\mes_{\A,j}'$. If \Adv' does not abort, it computes for all $i\in[n]\setminus\{i^*\}$
$$
\mes_{\A,i}=r_i\oplus\H((r_{\ell})_{\ell\in[n]\setminus\{i\}})=\mes_{\A,j}'
$$
where there is a $j\in[Q]$ such that the last equality holds. It also uses oracle $\O$ to query for all $i\in[n]\setminus\{i^*\}$ the $n-1$ corresponding keys $\key_i$ that match with the transcripts containing $\mes_{\A,i}$. $\D_k$ sets $\mes_{\B,i}:=\mes_{\B,j}'$ and $s_{\B,i}:=\key_i$. It creates $\mes_{\B,i^*}$ and $s_{\B,i^*}$ as \Adv' does. It sends $(\mes_{\B,i})_{i\in[n]}$ to \Adv to receive its output which it uses together with $(s_{\B,i})_{i\in[n]}$ as input for \D. $\D_k$ outputs \D's output.  
\begin{eqnarray*}
\epsilon_{k}&=&|\Pr[\D_k^{\O_{k}}(1^{\sec})]=1]-Pr[\D_k^{\O_{u}}(1^{\sec})=1]|\\
&= &\frac{1}{Q}|\Pr[\D((s_{\B,i})_{i\in[n]},\Adv)_{\D_k^{\O_{k}}})=1]-Pr[\D((\set_{\B,i^*},\Adv)_{\D_k^{\O_{u}}})=1]|\\
&\geq&\frac{1}{Q}\epsilon_{\D}.
\end{eqnarray*}

For the last step, we need to replace $s_{\B,i^*}$ with $s_{\A,i^*}$. We use the same argument as in Claim~\ref{claim:malsender} using the correctness of the scheme. Hence we obtain

\begin{eqnarray*}
\epsilon_{\OT} &=&|\Pr[\D((\send, \Adv)_{\Pi})=1] -\Pr[\D((\OOT^\E,\Adv))=1]|\\
&\leq & Q\epsilon_{u}+|\Pr[\D((s_{\B,i})_{i\in[n]},\Adv)=1]-\Pr[\D(\set_{\B,i^*}, \Adv)=1]|\\
&\leq & Q(\epsilon_{u}+\epsilon_{k})+|\Pr[\D(\set_{\B,i^*},\Adv)=1]-\Pr[\D(\set_{\A,i^*}, \Adv)=1]|\\
&\leq &Q(\epsilon_{u}+\epsilon_{k})+(1-\delta),
\end{eqnarray*}
where $\set_{\A,i^*}:=(s_{1},\dots, s_{i^*-1},s_{\A,i^*},s_{i^*+1},\dots, s_{n})$.
\pe
\end{proof}
\pe
\end{proof}
