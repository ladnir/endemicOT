\newcommand{\msng}{\textcolor{red}{(Missing References?)}}

\section{Introduction}

An oblivious transfer (OT) \cite{Rabin81,C:EveGolLem82} is a cryptographic primitive often used in the context of secure multi party computation, which allows to preserve the privacy during a joint computation. Among others, it solves the task of securely distributing cryptographic keys for garbled circuits, which can be seen as encrypted programs. The combination of garbled circuits and oblivious transfer gives a generic solution for securely computing any functionality between two parties \cite{FOCS:Yao82b,FOCS:Yao86,STOC:Kilian88,C:IshPraSah08,EC:IKOPS11} and multiple parties \cite{C:CreVanTap95,EC:BenLin18,EC:GarSri18a}.   


In an OT, a sender and a receiver interact in a protocol and at the end of the protocol, the sender outputs two messages $s_0$, $s_1$ while the receiver outputs $b, s_b$ for choice bit $b$. Security asks that the sender does not learn $b$ and the receiver does not learn $s_{1-b}$. It is known that an OT implies key exchange and can be constructed from special types of public key encryption (PKE) \cite{FOCS:GKMRV00,C:PeiVaiWat08,cryptoeprint:2018:473}, generalizations of dual mode PKE \cite{TCC:GarIshSri18} or certified trapdoor permutations \cite{C:OstRicSca15}. Though, all of these solutions come with some drawbacks when it comes to practical deployment. They either only achieve a weak security notion \cite{FOCS:GKMRV00} or lack efficiency due to a special type of commitment protocol \cite{STOC:Kilian92,C:OstRicSca15,cryptoeprint:2018:473} or dual-mode cryptosystem \cite{C:PeiVaiWat08, TCC:GarIshSri18}, which is less efficient than standard PKE and only known from DDH, QR and LWE with weaker parameter choices\footnote{Peikert et al. require SIVP hardness for approximation factor $\tilde{O}(n^{3.5})$ while Regev's PKE \cite{STOC:Regev05} only requires $\tilde{O}(n^{1.5})$.}. There exist also solutions tailored to specific assumptions. Naor and Pinkas \cite{SODA:NaoPin01} constructed OTs from the DDH assumption and Brakerski and D\"ottling \cite{TCC:BraDot18} from the LWE assumption that requires similar parameter choices as Peikert et al \cite{C:PeiVaiWat08}.

In practice, a common approach is to use a critical amount of OTs in the random oracle model (ROM) \cite{CCS:BelRog93} and then extend the amount of OTs to the desired amount of OTs using OT extension \cite{STOC:Beaver96a,C:IKNP03,RSA:OrrOrsSch17,EC:ALSZ15,C:KelOrsSch15}. A random oracle is an ideal hash function that usually is instantiated with a concrete hash function in the implementation. The ROM brings major efficiency improvements and is therefore very common for practical cryptographic constructions, even though it might bring potential security weaknesses \cite{STOC:CanGolHal98}. 

In the ROM, Bellare \& Micali \cite{C:BelMic89} constructed OT based on the CDH assumption. Chou \& Orlandi \cite{LC:ChoOrl15} claimed a more efficient OT construction which was proven with some caveats under the GapDH assumption \cite{cryptoeprint:2017:1011}. Hauck \& Loss improved the construction to base it on the CDH assumption \cite{cryptoeprint:2017:1011}. Barreto et al. \cite{EPRINT:BDDMN17b} constructed an CDH based OT in the global random oracle model.  

A drawback of the more efficient constructions of Chou \& Orlandi, Hauck \& Loss, and Barreto et al. is that they require three or more rounds. The former also suffers technical issues with the ability to extract the input of the receiver \cite{LC:ChoOrl15}. Further, unlike the more generic constructions based on PKE, they are tailored to specific assumptions. A more generic construction for OT in the ROM would be preferable since it allows an easier transition to different assumption like LWE or LPN, which unlike CDH and DDH are assumed to offer security in presence of quantum computers. In this work, we therefore want to focus on the question:

\begin{center}
\emph{How to construct a versatile, highly efficient and fully secure OT in the ROM?}
\end{center}

For efficiency, we ask for a minimal round complexity, low computational complexity and compatibility with OT extension techniques.

\subsection{Our Contribution}

We start with a basic security definition which has previously been considered by Garg, Ishai and Srinivasan \cite{TCC:GarIshSri18} as OT correlations functionality. We call this security \emph{endemic} security and denote an OT that is endemically secure with endemic OT. Endemic security allows to achieve a minimal round complexity, also denoted as non-interactive OT \cite{C:BelMic89,TCC:GarIshSri18} as well as analyze the security of optimized existing OT and OT extension protocols.

In \sectionref{sec:relot}, we compare endemic security notion with other notions and show that an endemic OT can efficiently  be transformed such the other considered security notions are achieved but potentially at the cost of a higher round complexity. We also show that  only endemic security permits a one-round or non-interactive OT.

In \sectionref{sec:endemicOT}, we give a construction in the ROM that transforms any two message key agreement protocol, where the distribution of one of the messages is computationally close to uniform, into an endemically secure two message $1$-out-of-$n$ oblivious transfer\footnote{
\iffullversion 
In \namedref{Appendix}{sec:allbutone}, we show how the framework can be adapted to obtain an endemically secure $(n-1)$-out-of-$n$ OT.
\else 
In the full version, we show how the framework can be adapted to obtain an endemically secure $(n-1)$-out-of-$n$ OT.
\fi
}. Further, 
if the key agreement protocol is a one-round protocol, we obtain a one-round endemic OT. This implies that we get a one-round endemic OT from DDH, CDH and two round endemic OT from LWE, LPN, McEliece and Subset Sum. Note that \cite{TCC:GarIshSri18} have a one-round OT from LWE, DDH, QR in the CRS model.

In \sectionref{sec:otext}, we show that our construction is compatible with OT extension techniques. Concretely, we show that endemic OTs can be extended to a larger amount of endemic OTs using only one additional round. This allows us to obtain $\mathsf{poly}(\sec)$ OTs using only $O(\sec)$ public key operations in only two rounds. We revisit the OT extension protocol of Keller et al., Orr{\`u} et al. and Asharov et al. \cite{C:KelOrsSch15,RSA:OrrOrsSch17,JC:ALSZ17} under endemic security. It turns out that its uniform message security can be fully broken. We point out attacks and provide fixes such that classical, uniform and endemic security can be obtained. Finally, we observe that most OT extension protocols are implemented \cite{libOTe,KOS,EMP} using an ideal cipher in place of a random oracle. However, these implementations have no security proofs and we show that they too can be fully broken. We give new protocols and proofs in the ideal cipher model which allows a 10 times speed up on the ROM when implemented.  

In \sectionref{sec:impl}, we implement our construction based on the Diffie-Hellman key exchange and the Module LWE (\MLWE) based Kyber key encapsulation \cite{NISTPQC-R1:CRYSTALS-KYBER17}. We emphasize that it can also be instantiated with many of the other NIST post-quantum standardisation candidates and is to the best of our knowledge the first implementation of a quantum resistant OT.

\subsection{Our Techniques}

\paragraph{Endemic Security.} When defining malicious security of an OT, one defines an ideal functionality $\OOT$. An OT is called secure, if for any adversary against the OT scheme, there exists an adversary interacting with \OOT producing the same output. Classically, $\OOT$ either receives the OT strings $s_0$, $s_1$ as input from the sender or samples them uniformly at random and outputs them to the sender. But there are also OTs where the receiver can determine the OT strings or even both parties could influence how the OT strings are generated. We distinguish four main security notions.
\begin{description}
\item[Uniform Message Security:] The ideal functionality $\OOT^{\U}$ samples the OT strings uniformly at random and outputs the strings to sender and receiver.
\item[Sender Chosen Message Security:] The ideal functionality $\OOT^{\send}$ receives the OT strings from the sender and outputs one of the strings to the receiver.
\item[Receiver Chosen Message Security:] The ideal functionality $\OOT^{\rec}$ receives one of the OT strings from the receiver, samples the other one uniformly at random and outputs the strings to the sender.
\item[Endemic Security] If the sender is malicious, it chooses both strings. If the receiver is malicious, it chooses one of the strings. All strings that are not choosen yet, are sampled uniformly at random by the ideal functionality $\OOT^{\E}$ and forwarded the sender and receiver such that the sender has both strings and the receiver has one of the strings. 
\end{description}

Notice that endemic security gives the weakest security guarantees, no matter whether the receiver or the sender is malicious, the malicious party can always determine the output distribution. Uniform message security gives very strong security guarantees since a malicious party can never change the distribution to non-uniform.
     

\paragraph{Relations Between Security Notions.} We show on one hand that an OT with uniform message security is also secure with respect to all other security notions. On the other hand, uniform, sender and receiver chosen message security imply endemic security. Still, there are very simple transformations from an endemically secure OT to an OT that achieves any of the other security notions. Though we remark that uniform message security implies and therefore requires a secure coin tossing protocol. In \figureref{fig:OTrelations}
\begin{figure}
\centering
\begin{tikzpicture}
\node (U) at (4,4) {Uniform Message Security};
\node (SC) at (0,2) {$\OOT^\send$-Security};
\node (RC) at (8,2) {$\OOT^\rec$-Security};
\node (E) at (4,0) {Endemic Security};
\node (TC) at (3,2) {Coin Tossing};
\draw [-implies,double equal sign distance] (U) -- (SC);
\draw [-implies,double equal sign distance] (U) -- (RC);
\draw [-implies,double equal sign distance] (SC) -- (E);
\draw [-implies,double equal sign distance] (RC) -- (E);
\draw [->, thick] (U) to [out=240,in=90] (TC);
\draw [-, thick] (E) to [out=60,in=270] (5,2);
\draw [->, thick] (5,2) to [out=90,in=300] (U);
\draw [-, thick] (TC) to [out=0,in=300] (4.655,3);
\draw [->, thick] (E) to [out=180,in=270] (SC);
\draw [->, thick] (E) to [out=0,in=270] (RC);
\end{tikzpicture}
\label{fig:OTrelations}
\caption{
The figure depicts the different security notions of OT and their relations. $A\Rightarrow B$ denotes that security $A$ implies security $B$. $A\rightarrow B$ denotes that any OT realizing security $A$ can be efficiently transformed into an OT realizing security $B$.
}
\end{figure}
we give an overview over these implications and transformations. 

We also show that endemic OT is weaker than the other notions but at the same time this allows a minimal round complexity of a single round. More precisely, we show that there is no one round OT that achieves sender or receiver chosen message security.

\paragraph{From Key Agreement to OT.}

point out solutions in crs model, kilian commitment and hashing technique

\paragraph{Instantiation and Implementation.}

\paragraph{Secure Employment of OT Extension.}
