\section{Instantiations}\label{sec:inst}
In the following, we first show how to efficiently instantiate the construction in \figureref{fig:KAtoOT} using the Diffie-Hellman key exchange. In particular, we show how a tighter security reduction can be obtained using the random self-reducibility of the DDH assumption. In \figureref{fig:optKAtoOT}, we give an optimized variant based on an interactive DDH assumption.

Afterwards, we show how to instantiate the construction in \figureref{fig:KAtoOT} based on the lattice based Kyber key agreement. 

We emphasize that the instantiations only achieve stand alone security. For UC security, one needs to assume that DDH and Kyber are secure against non-uniform adversaries. 


\subsection{Instantiation from DDH}



\begin{definition}[$n$-Multi-Instance DDH Assumption]
For a group $\G$, the \emph{decisional Diffie-Hellman} assumption is hard if for any ppt distinguisher \D,
$$
|\Pr[ \D(\lb 1\rb,\lb \vec{a}\rb,\lb b\rb,\lb \vec{a}b\rb)=1]-\Pr[ D(\lb 1\rb,\lb \vec{a}\rb,\lb b\rb,\lb \vec{c}\rb)=1]|=\negl,
$$
where $\vec{a}\leftarrow\Z_p^n$, $b\leftarrow\Z_p$ and $\vec{c}\leftarrow\Z_p^n$.
\end{definition}

In Appendix~\ref{sec:DDH}, we show that $n$-multi-instance DDH is secure under the DDH assumption with a security loss of $n$. In the following we show that the Diffie-Hellman key exchange is tightly multi-instance secure under multi-instance DDH. 

\begin{lemma}\label{lem:DDH}
Let $Q$ and $n$ be polynomial in $\sec$.
The Diffie-Hellman key exchange over \G is unconditionally $Q$-multi-instance uniform. Further, let the $n$-multi-instance DDH assumption hold over group \G except with advantage $\epsilon$, then the Diffie-Hellman key exchange is one-round $(Q,n)$-multi-instance key-indistinguishable except advantage $\epsilon-\negl$. 
\end{lemma}

\begin{proof}
The distribution of $\lb a\rb$ over \G is uniform, therefore 
$$
|\Pr[ \D^{\O_{\A}}(1^{\sec})\rb=1]-Pr[ \D^{\O_{u}}(1^{\sec})=1]|=0,
$$
even against an unbounded $\D$. Hence, the Diffie-Hellman key exchange is unconditional $Q$-multi-instance uniform.

For proving the second part of the lemma, we construct a ppt distinguisher \D that breaks $n$-multi-instance DDH assumption given a ppt distinguisher $\D_\key$ that breaks the $(Q,n)$-multi-instance key-indistinguishability of the Diffie-Hellman key exchange. $\D$ receives a challenge $\lb \vec{a}\rb,\lb b\rb,\lb \vec{c}\rb$, sets $\mes_{\B}:=\lb b\rb$ and invokes $\D_{\key}$ on input $\mes_{\B}$. On the $j$-th query of $\D_{\key}$ to $\O_{\A}$, \D samples $\vec{r}_j\leftarrow\Z_p^{n+1}$ and responds with $\mes_{\A,j}:=\lb \langle (\vec{a},1),\vec{r}_j\rangle\rb=\lb a_1\rb^{r_{j,1}}\cdot\lb a_2\rb^{r_{j,2}}\ldots\lb a_n\rb^{r_{j,n}}\cdot\lb r_{j,n+1}\rb$. When $\D_{\key}$ queries $\O_{\key}$ for key $\key_j$, \D responds with $\key_j:=\lb \langle\vec{c},\vec{r}_j\rangle\rb=\lb c_1\rb^{r_{j,1}}\cdot\lb c_2\rb^{r_{j,2}}\ldots\lb c_n\rb^{r_{j,n}}\cdot\lb b\rb^{r_{j,n+1}}$. In the end, $\D$ outputs the output of $\D_{\key}$.

It is easy to see that $\O_{\A}$ has the correct output distribution. $r_{j,n+1}$ is uniform over $\Z_p$ and hence $\mes_{\A,j}$ is. Further, conditioned on $\mes_{\A,j}$, $r_{j,1},\ldots,r_{j,n}$ are uniform. Given that $\vec{c}=\vec{a}b$, the output
$$
\key_j=\lb \langle\vec{c},\vec{r}_j\rangle\rb\cdot\lb b\rb^{r_{j,n+1}}=\lb \langle\vec{a}b,\vec{r}_j\rangle\rb\cdot\lb b\rb^{r_{j,n+1}}=\lb \langle(\vec{a},1),\vec{r}_j\rangle\rb^b=\mes_{\A,j}^b
$$
of $\O_{\key}$ is also distributed correctly. In case that $\vec{c}$ is uniform, we need to show that all the $n$ outputs of $\O_{\key}$, $\vec{\key}=\key_1,\ldots,\key_n$ are uniform. Let $\mes_i$ be the message $\mes_{\A,j}$ and $\vec{t}_i$ the randomness $\vec{r}_j$ that corresponds to $\key_i$. Since $\vec{c}$ is uniform and $\vec{\key}=\lb \vec{c}\cdot T\rb$, where $T$ is the matrix with $i$-th column $\vec{t}_i$, $\vec{\key}$ is uniform if $T$ is invertible. Since $r_{j,1},\ldots,r_{j,n}$ are uniform given $\mes_{\A,j}$ so is $T$. For a uniform $T$ over $\Z_p^{n\times n}$, the probability that $T$ is invertible is that all the rows are linear independent, i.e. 
$$
\Pr[ T\text{ invertible}]= \frac{1}{p^{n^2}}\prod_{i=0}^{n-1}(p^n-p^i)\geq\left(1-\frac{1}{p}\right)^n\geq 1-\negl.
$$ 
Therefore, except with negligible probability, $\D$ has the same advantage in breaking $n$-multi-instance DDH as $\D_{\key}$ has in breaking one-round $(Q,n)$-multi-instance key-indistinguishability.
\pe
\end{proof}

Using Theorem~\ref{thm:KAtoOT}, Lemma~\ref{lem:DHuniform} and Lemma~\ref{lem:DDH}, we obtain the following corollary.

\begin{corollary}
When instantiating an $1$ out of $n$ OT in \figureref{fig:KAtoOT} with Diffie-Hellman key exchange over group \G, then in the programmable random oracle model the resulting endemic OT is statistically secure against malicious senders and secure against malicious receivers except advantage $(n-1)Q\epsilon_{\DDH}+\negl$, where the DDH assumption over group $\G$ holds except advantage $\epsilon_{\DDH}$ and $Q$ is a bound on the amount of adversarial random oracle queries.
\end{corollary}

\begin{remark}
If we apply a hardcore predicate to $\key_{\A}$ and $\key_{\B}$ in the protocol in \figureref{fig:DH} and apply the transformation \figureref{fig:KAtoOT}, we receive a one-round endemically secure OT in the random oracle model based on the computational Diffie-Hellman assumption. Alternatively, one could also use the random oracle instead of a hardcore predicate to obtain longer OT strings. 
\end{remark}

\subsection{Optimized, Interactive DDH based Instantiation}\label{sec:optOT}
\begin{figure}
\centering
\framebox{
\begin{tikzpicture}[xscale=1.3]
\node [anchor=west] at (-0.5,5) {Sender:};
\node [anchor=west] at (4,5) {Receiver$(\i\in[n])$:};
\node [anchor=west] at (4,4.5){$\forall j\in[n]\setminus\{\i\}: r_{j}\leftarrow\G$};
\node [anchor=west] at (4,4){$a\leftarrow\Z_p$};
\node [anchor=west] at (4,3.5){$r_{\i}=\lb a\rb \ominus \H((r_j)_{j\neq\i})$};
\draw [thick, <-] (2.75,3.5)-- node [midway,above]{$(r_j)_{j\in[n]}$}(4,3.5);
%\node [anchor=west] at (-0.35,2.5){$\mes_{\A,j}=r_j\oplus h_j$};
\node [anchor=west] at (-0.5,4){$b\leftarrow\Z_p$};
\draw [thick, ->] (2.75,3.35)-- node [midway,below]{$\lb b \rb$}(4,3.35);
\node [anchor=west] at (4,3) {$s_{\A,\i}=\lb ab\rb$};
\node [anchor=west] at (-0.5,3.5) {$\forall j\in[n]$};
\node [anchor=west] at (-0.35,3) {$s_{\B,j}=(r_j\H((r_\ell)_{\ell\neq j}))^b$};
\end{tikzpicture} 
}
\myvspace{-0.3cm}
\caption{The figure shows an optimized variant of the protocol from \figureref{fig:KAtoOT} based on an interactive DDH assumption.}
\label{fig:optKAtoOT}
\end{figure}
In \figureref{fig:optKAtoOT}, we show an optimized variant of our OT. It reduces the communication cost from the sender to the receiver by sending only a single group element. This is possible since it does not depend on any of the $n$ elements sent by the receiver. A drawback of this construction is that we do not know how to prove its security under the standard DDH assumption.

The reason is simple, in our security proof during the simulation, $\Adv'$ needs compute the key $\lb ab \rb$ while at the same time given $\lb 1\rb,\lb a'\rb,\lb b\rb$, $\lb ab'\rb$ needs to be hard to distinguish from a uniformly random group element. Since $a$ is picked by the adversary $\Adv$ and only transmits $\lb a \rb$, we do not know how to simulate correctly. The next definition formally states the assumption under which the protocol in \figureref{fig:optKAtoOT} can be proven to be secure. 

\begin{definition}[Interactive Decisional Diffie-Hellman (IDDH) Assumption]\label{def:IDDH}
For a group $\G$, the \emph{interactive decisional Diffie-Hellman} assumption is hard if for any ppt distinguisher $\D_1,\D_2$,
\begin{eqnarray*}
&&|\Pr[ \D_2(\mathsf{st},\lb 1\rb,\lb a\rb,\lb b\rb,\lb xb\rb,\lb ab\rb)=1]\\
&&-\Pr[ \D_2(\mathsf{st},\lb 1\rb,\lb a\rb,\lb b\rb,\lb xb\rb,\lb c\rb)=1]|=\negl,
\end{eqnarray*}
where $a\leftarrow\Z_p$, $b\leftarrow\Z_p$ and $c\leftarrow\Z_p$ and $(\mathsf{st},\lb x\rb)\leftarrow\D_1(\lb 1\rb)$.
\end{definition}

\subsection{Instantiation based on Crystals-Kyber}
 

\begin{figure}[h!]
\centering
\framebox{
\begin{tikzpicture}[scale=1.3]
\node [anchor=west] at (-0.25,5) {\A:};
\node [anchor=west] at (4.7,5) {\B:};
\node [anchor=west] at (-0.25,4.5){$(\sk,\mes_\A)\leftarrow\kybergen()$};
\draw [thick, ->] (3.6,4.5)-- node [midway,above]{$\mes_\A$}(4.75,4.5);
\node [anchor=west] at (4.7,4.5){$\key_{\B}\leftarrow\kB^{32}$};
\node [anchor=west] at (4.7,4){$\mes_\B=\kyberenc(\pk, \key_{\B})$};
\draw [thick, <-] (3.6,4)-- node [midway,above]{$\mes_\B$}(4.75,4);
\node [anchor=west] at (-0.25,3.5) {$\key_\A=\kyberdec(\sk,\mes_\B)$};
\node [anchor=west] at (4.7,3.5) {$\key_\B$};
\end{tikzpicture}
}
\label{fig:Kyber}
\caption{The figure shows a Kyber based key agreement protocol between parties \A and \B.}
\end{figure}

\begin{definition}[Crystals-Kyber CPAPKE]
\emph{ Crystals-Kyber CPAPKE} is a correct and CPA secure public key encryption based on the Module LWE (\MLWE) assumption. We follow the specifications of Kyber, in which $\kB$ denotes the  set $\{0,1,\ldots,255\}$. Kyber is parameterized by parameters $d_t=11$, $\nLWE=256$, $\kLWE\in\{2,3,4\}$, $q=7681$ and ring $\R_q=\Z_q[ X]/(X^{\nLWE}+1)$. 

$(\kybergen,\kyberenc,\kyberdec)$ have the following syntax. 
\begin{description}
\item[$\kybergen$:] Outputs a secret and public key pair $\pk,\sk$, where $\pk=(\cdt,\rho)\in\kB^{\frac{\kLWE\nLWE d_t}{8}}\times\kB^{32}$.
\item[$\kyberenc$:] Takes as input a public key $\pk$, a message $\mes\in\kB^{32}$ and random coins $\tape\in\kB^{32}$. It outputs a ciphertext $\kyberc$.
\item[$\kyberdec$:] Takes as input secret key $\sk$ and a ciphertext $\kyberc$. It outputs a message $\mes$.
\end{description}
Further, Crystals-Kyber specifies the following algorithms.
\begin{description}
\item[$\kdec_\ell$:] Takes a an element in $\kB^{32\ell}$ and maps it to $\R_q$. The inverse operation is $\kenc_\ell$.
\item[$\kcom_q(*,d)$:] Takes a an element in $\R_q$ and maps it to a polynomial with coefficients in $\Z_{2^d}$. The inverse is $\kdecom_q(*,d)$.
\item[$\kparse$:] Takes a uniform byte stream in $\kB^{*}$ and maps it to a uniform element in $\R_q$.
\end{description}
\end{definition}

It is important to know, that $\cdt:=\kenc_{d_t}(\kcom_q(\mt, d_t)$, where $\mt$ is computationally indistinguishable from a uniform element in $\R_q^{\kLWE}$ based on the \MLWE assumption. In the following, we will choose $d_t=13$ such that $q<2^{d_t}$ and no compression takes place. This will help us to avoid complications and does not decrease efficiency besides a slightly larger public key $\pk$. Further, we will keep component $\rho$ of $\pk$ consistent between all $\pk$ used in a single $1$ out of $n$ OT.

In order to instantiate our framework, we need to define a group operation $\pk\oplus\pk$ and a hash functions that maps to a uniform $\cdt$ component of $\pk$, i.e. a uniform element in $\R_q^{\kLWE}$. For the latter, we use a hash function that produces an output bit stream, which is in $\kB^{*}$, and use $\kLWE$ different parts of the stream and apply $\kparse$ to the $\kLWE$ streams to obtain a (pseudo) uniform element in $\R_q^{\kLWE}$ whenever the bit streams are (pseudo) uniform.

We define the group operation $\pk_1\oplus\pk_2$, by mapping $\pk_1=(\cdt_1,\rho)$, $\pk_2=(\cdt_2,\rho)$ to $\pk_3=(\cdt_3, \rho)$, where $\cdt_3=\kenc_{13}(\kdec_{13}(\cdt_1)+\kdec_{13}(\cdt_2))$ and $+$ is the addition in $\R_q^{\kLWE}$. $\ominus$ is defined correspondingly. 

By using \theoremref{thm:KAtoOT}, \lemmaref{lem:multuniform} and \lemmaref{lem:keytomultkey}, we get the following corollary.
 
\begin{corollary}
When instantiating an $1$ out of $n$ OT in \figureref{fig:Kyber} with Crystals-Kyber, then in the programmable random oracle model the resulting endemic OT is secure against a malicious sender except advantage $(n-1)\epsilon_{\MLWE}+\negl$  and secure against malicious receivers except advantage $Q^2\epsilon_{\MLWE}+\negl$, where the \MLWE assumption holds except advantage $\epsilon_{\MLWE}$ and $Q$ is a bound on the amount of adversarial random oracle queries.
\end{corollary}

In this work, we will instantiate Kyber with $k=3$ which is claimed to have a qbit security level of $161$ bit. This security level does not immediately carry over to our Kyber based OT, there is an additional security loss of $Q^2$. Though we are unaware of an attack that is significantly more efficient on our Kyber based OT than the attacks on Kyber.