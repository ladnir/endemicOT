\section{UC Security against Malicious Receivers}\label{sec:UCOT}

In \theoremref{thm:KAtoOT} in \sectionref{sec:endemicOT}, we only show stand-alone security. While UC security for a malicious sender is already covered by \claimref{claim:malsender}, 
\claimref{claim:malreceiver} uses an adversary \Adv' that rewinds \Adv and hence does not accomplish UC security. Here, we proof UC security against a malicious receiver for two settings. First, in case of a two-round OT from any uniform two-round key agreement. Second, in case of a one-round OT based on the Diffie Hellman key agreement under a variant of \DDH.  


\subsection{UC Security of the Two-Round OT}
 
\begin{claim}\label{claim:UCmalreceiver}
Given a $\delta$ correct,  $Q$-multi-instance $\epsilon_u$-uniform,  $(Q,n-1)$-multi-instance $\epsilon_k$-key indistinguishable two-round \UKA scheme, where $Q$ upper bounds the amount of random oracle queries by an adversary. Then the proposed two-round OT is UC-secure against malicious receivers, i.e. in the programmable random oracle model for any ppt adversary \Adv, there exists a ppt adversary \Adv' such that for any ppt distinguisher \D and any polynomial size auxiliary input $z$,
$$
|\Pr[\D(z,(\send, \Adv)_{\Pi})=1] -\Pr[\D(z,(\OOT^\E,\Adv'))=1]|\leq 2Q\epsilon_{u}+Q\epsilon_{k}+(1-\delta),
$$
where all algorithms receive input $1^\sec$.
\end{claim}


\begin{proof}
We follow the same line of arguement as in \claimref{claim:malreceiver} with the exception that we now use a hybrid argument rather than guessing the correct random oracle query. 

A malicious receiver will make random oracle queries and each query will correspond to a potential choice for $(r_i)_{i\in[n]}$. For each potential choice of $(r_i)_{i\in[n]}$ there will be corresponding OT strings, i.e. keys computed by the key agreement.
 
During the first hybrid, we replace the corresponding keys of the first random oracle query with uniform. Through a sequence of hybrids, we will do this for every query until all the keys are replaced with uniform. Notice that in each hybrid, $\D$ will only get to see the key that corresponds to the malicious receivers choice of $(r_i)_{i\in[n]}$ and not for all potential choices of $(r_i)_{i\in[n]}$.


We start by giving a description of \Adv'. For each random oracle query $q$ to $\H_i$ for an $i\in[n]$, \Adv' responds with a random group element $\H_i(q)\leftarrow\G$.
 When \Adv sends $(r_i)_{i\in[n]}$, \Adv' looks up the first oracle query of the form $q= (r_1,\dots, r_{i^*-1},r_{i^*+1},\dots, r_{n})$ for an $i^*\in[n]$. \Adv' sends $i^*$ to $\OOT^\E$. \Adv' computes for all $i\in[n]$ $\mes_{\A,i}:=r_i\oplus\H_i((r_\ell)_{\ell\neq i})$, $\tape_{\B,i}\leftarrow\bits^*$ and $\mes_{\B,i}\leftarrow\B(\tape_{\B,i},\mes_{\A,i})$. It also computes $s_{\B,i^*}:=\EK(\tape_{\B,i^*},\mes_{\A,i^*})$.
\Adv' sends $s_{\B,i^*}$ to $\OOT^\E$, $(\mes_{\B,i})_{i\in[n]}$ to \Adv and outputs the output of \Adv. This concludes the description of \Adv'. We emphasize that here, the other OT strings $(s_{\B,i})_{i\neq i^*}$ will not be the output of $\EK(\tape_{\B,i},\mes_{\A,i})$ but sampled uniformly by $\OOT^\E$.

We now define a sequence of hybrids. The first hybrid is $\hyb_1$ and corresponds the interaction of \Adv with the sender of the protocol description. The last hybrid, $\hyb_{3(Q+1)}$, corresponds to $\Adv'$.
For $k\in[Q+1]$, we define:
\begin{description}
\item[$\hyb_{3k}$:]  When \Adv makes an oracle query $q$ respond with a random group element $\H_i(q)\leftarrow\G$. 

When \Adv sends $(r_i)_{i\in[n]}$, look up the first oracle query of the form $q= (r_1,\dots, r_{d-1},r_{d+1},\dots, r_{n})$ to $\H_d$ for a $d\in[n]$. Let $j$ be the query number. Compute for all $i\in[n]$, $\mes_{\A,i}:=r_i\oplus\H_i((r_\ell)_{\ell\neq i})$, $\tape_{\B,i}\leftarrow\bits^*$ and $\mes_{\B,i}\leftarrow\B(\tape_{\B,i},\mes_{\A,i})$. Further, compute $s_{\B,i}:=\EK(\tape_{\B,i},\mes_{\A,i})$. If $j<k$, sample for all $i\neq d$, $s_i$ uniformly. Otherwise, for all $i\neq d$, $s_i:=s_{\B,i}$.   Define $\set_{\B,d}:=(s_{1},\dots, s_{d-1},s_{\B,d},s_{d+1},\dots, s_{n})$.  
 Send $(\mes_{\B,i})_{i\in[n]}$ to \Adv and output $\set_{\B,d}$ together with the output of \Adv. 
\item[$\hyb_{3k+1}$:]  \textcolor{red}{When \Adv makes an oracle query $q$ to $\H_i$ for an $i\in[n]$ and the query number is less or equal to $k$, respond with a random group element $\H_i(q)\leftarrow\G$. 
If the query number equals $k$, store $i^*:=i$ and $(g^*_1,\dots, g^*_{i^*-1},g^*_{i^*+1},\dots, g^*_{n}):=q$. 
For all following random oracle queries, i.e. the query number $j$ is higher than $k$, respond with a random group element $\H_i(q)\leftarrow\G$ if $i=i^*$ or for all $g\in\G$ $q_j\neq (g^*_1,\dots, g^*_{i^*-1},g,g^*_{i^*+1},\dots, g^*_{n})\setminus g^*_i$. Otherwise sample random tape $\tape_j\leftarrow\bits^*$ and compute $\mes_{j}\leftarrow\A(\tape_j)$. 
Respond with $\H_i(q_j):=\mes_j\ominus g^*_{i}$.}

When \Adv sends $(r_i)_{i\in[n]}$, look up the first oracle query of the form $q= (r_1,\dots, r_{d-1},r_{d+1},\dots, r_{n})$ to $\H_d$ for a $d\in[n]$. Let $j$ be the query number. Compute for all $i\in[n]$, $\mes_{\A,i}:=r_i\oplus\H_i((r_\ell)_{\ell\neq i})$, $\tape_{\B,i}\leftarrow\bits^*$ and $\mes_{\B,i}\leftarrow\B(\tape_{\B,i},\mes_{\A,i})$. Further, compute $s_{\B,i}:=\EK(\tape_{\B,i},\mes_{\A,i})$. If $j<k$, sample for all $i\neq d$, $s_i$ uniformly. Otherwise, for all $i\neq d$, $s_i:=s_{\B,i}$.   Define $\set_{\B,d}:=(s_{1},\dots, s_{d-1},s_{\B,d},s_{d+1},\dots, s_{n})$.  
 Send $(\mes_{\B,i})_{i\in[n]}$ to \Adv and output $\set_{\B,d}$ together with the output of \Adv. 
 \item[$\hyb_{3k+2}$:]  When \Adv makes an oracle query $q$ to $\H_i$ for an $i\in[n]$ and the query number is less or equal to $k$, Respond with a random group element $\H_i(q)\leftarrow\G$. 
If the query number equals $k$, store $i^*:=i$ and $(g^*_1,\dots, g^*_{i^*-1},g^*_{i^*+1},\dots, g^*_{n}):=q$. 
For all following random oracle queries, i.e. the query number $j$ is higher than $k$, respond with a random group element $\H_i(q)\leftarrow\G$ if $i=i^*$ or for all $g\in\G$ $q_j\neq (g^*_1,\dots, g^*_{i^*-1},g,g^*_{i^*+1},\dots, g^*_{n})\setminus g^*_i$. Otherwise sample random tape $\tape_j\leftarrow\bits^*$ and compute $\mes_{j}\leftarrow\A(\tape_j)$. 
Respond with $\H_i(q_j):=\mes_j\ominus g^*_{i}$. 

When \Adv sends $(r_i)_{i\in[n]}$, look up the first oracle query of the form $q= (r_1,\dots, r_{d-1},r_{d+1},\dots, r_{n})$ to $\H_d$ for a $d\in[n]$. Let $j$ be the query number. Compute for all $i\in[n]$, $\mes_{\A,i}:=r_i\oplus\H_i((r_\ell)_{\ell\neq i})$, $\tape_{\B,i}\leftarrow\bits^*$ and $\mes_{\B,i}\leftarrow\B(\tape_{\B,i},\mes_{\A,i})$. Further, compute $s_{\B,i}:=\EK(\tape_{\B,i},\mes_{\A,i})$. \textcolor{red}{If $j\leq k$,} sample for all $i\neq d$, $s_i$ uniformly. Otherwise, for all $i\neq d$, $s_i:=s_{\B,i}$.   Define $\set_{\B,d}:=(s_{1},\dots, s_{d-1},s_{\B,d},s_{d+1},\dots, s_{n})$.  
 Send $(\mes_{\B,i})_{i\in[n]}$ to \Adv and output $\set_{\B,d}$ together with the output of \Adv. 
\end{description}

\begin{claim}\label{claim:first}
For any $k\in[Q+1]$, let there be a distinguisher $\D$ with
$$
\epsilon_{\D}:=|\Pr[\D(z,\hyb_{3k})=1] -\Pr[\D(z,\hyb_{3k+1})=1]|.
$$
Then, there is a distinguisher $\D_u$ breaking the $Q$-multi-instance uniformity of the \UKA protocol.
\end{claim}

\begin{proof}
 $\D_u$ gets access to an oracle $\O$ which either outputs uniform messages, i.e. $\O_u$ or messages of  the form $\mes_\A\leftarrow\A(\tape_\A)$ for $\tape_\A\leftarrow\bits^*$. $\D_u$ invokes \D and creates its input as follows. It invokes $\Adv$ and interacts with him as $\hyb_{3k+1}$ does with the difference that $\mes_j$ are requested from $\O$ rather than computing them.  After receiving the output, $\D_u$ uses it as input for $\D$ together with $(s_{\B,i})_{i\in[n]}$, where $s_{\B,i}\leftarrow\EK(\tape_{\B,i},\mes_{\A,i})$. $\D_u$ outputs the output of \D. 

If $\O$ is oracle $\O_{u}$, all $\mes_j$ are uniform and hence all random oracle queries $q$ are answered with a uniformly random $\H_i(q)\in\G$.  Otherwise, \Adv' is identical with \send as well as $(s_{\B,i})_{i\in[n]}$ are identical with the output of \send.  
Hence 
\begin{eqnarray*}
\epsilon_{u}&=&|\Pr[\D_u^{\O_{\A}}(z)]=1]-Pr[\D_u^{\O_{u}}(z)=1]|\\
&= &|\Pr[\D(z,((s_{\B,i})_{i\in[n]},\Adv)_{\D_u^{\O_{\A}}})=1]\\
&&-\Pr[\D(z,((s_{\B,i})_{i\in[n]},\Adv)_{\D_u^{\O_{u}}})=1]|\\
&\geq&\epsilon_{\D}.
\end{eqnarray*}
\pe
\end{proof}


\begin{claim}
For any $k\in[Q+1]$, let there be a distinguisher $\D$ with
$$
\epsilon_{\D}:=|\Pr[\D(z,\hyb_{3k+1})=1] -\Pr[\D(z,\hyb_{3k+2})=1]|.
$$
Then there is a distinguisher $\D_k$ that breaks the $(Q,n-1)$-multi-instance key-indistinguishability of the \UKA protocol.
\end{claim}

\begin{proof}
$\D_k$ has access to oracles $\O_{\langle \A,\B\rangle}$ and \O which is either $\O_u$ or $\O_k$. $\D_k$ invokes $\D$ and creates its input as follows. $\D_k$ invokes \Adv and interacts with it as $\hyb_{3k+2}$ does with the difference, that $\D_k$ generates $\mes_j$ by querying a transcript $\langle \A, \B\rangle=(\mes_{\A,j}',\mes_{\B,j}')$ from $\O_{\langle \A,\B\rangle}$ and setting $\mes_j=\mes_{\A,j}'$. 

If $(r_i)_{i\in[n]}$ corresponds to a query $j\neq k$, then $\hyb_{3k+1}$ and $\hyb_{3k+2}$ are equivalent. Follow the description of $\hyb_{3k+1}$ and ignore oracle $\O$, since the keys for the challenge transcripts are not needed.

If $(r_i)_{i\in[n]}$ corresponds to query $k$,
compute for all $i\in[n]\setminus\{i^*\}$
$$
\mes_{\A,i}:=r_i\oplus\H_i((r_{\ell})_{\ell\neq i})=\mes_{\A,j}'
$$
where there exists a $j\in[Q]$ such that the last equality holds. It also uses oracle $\O$ to query for all $i\in[n]\setminus\{i^*\}$ the $n-1$ corresponding keys $\key_i$ that match with the transcripts containing $\mes_{\A,i}$. $\D_k$ sets $\mes_{\B,i}:=\mes_{\B,j}'$ and $s_{\B,i}:=\key_i$. It creates $\mes_{\B,i^*}$ and $s_{\B,i^*}$ as usual. It sends $(\mes_{\B,i})_{i\in[n]}$ to \Adv to receive its output which it uses together with $(s_{\B,i})_{i\in[n]}$ as input for \D. $\D_k$ outputs \D's output.  
\begin{eqnarray*}
\epsilon_{k}&=&|\Pr[\D_k^{\O_{k}}(z)]=1]-Pr[\D_k^{\O_{u}}(z)=1]|\\
&= &|\Pr[\D(z,((s_{\B,i})_{i\in[n]},\Adv)_{\D_k^{\O_{k}}})=1]\\
&&-Pr[\D(z,(\set_{\B,i^*},\Adv)_{\D_k^{\O_{u}}})=1]|\\
&\geq&\epsilon_{\D}.
\end{eqnarray*}
\pe
\end{proof}

\begin{claim}\label{claim:first}
For any $k\in\{0\}\cup[Q]$, let there be a distinguisher $\D$ with
$$
\epsilon_{\D}:=|\Pr[\D(z,\hyb_{3k+2})=1] -\Pr[\D(z,\hyb_{3k+3})=1]|.
$$
Then, there is a distinguisher $\D_u$ breaking the $Q$-multi-instance uniformity of the \UKA protocol.
\end{claim}

\begin{proof}
The proof is almost identical to the proof of \claimref{claim:first} and therefore omitted.
\pe
\end{proof}

For the last step, we need to replace $s_{\B,i^*}$ with $s_{\A,i^*}$. We use the same argument as in Claim~\ref{claim:malsender} using the correctness of the scheme. Hence we obtain

\begin{eqnarray*}
\epsilon_{\OT} &=&|\Pr[\D(z,(\send, \Adv)_{\Pi})=1] -\Pr[\D(z,(\OOT^\E,\Adv))=1]|\\
&\leq &2Q\epsilon_{u}+Q\epsilon_{k}+(1-\delta).
\end{eqnarray*}
\pe
\end{proof}


\begin{remark}
For stand alone security, security of \UKA against uniform adversaries is sufficient, i.e. auxiliary input $z$ is the empty string. Further, for UC security in the global random oracle model, it is sufficient for the sender to send a salt at the start of each session that is used as an additional input to the random oracle within the session. 
\end{remark}







\subsection{Diffie-Hellman based One-Round Endemic OT with UC Security }\label{sec:ddhProof}



\begin{definition}[Choose-and-Open Decisional Diffie-Hellman (CODDH) Assumption]\label{def:CODDH}
	For a group $\G$, the \emph{choose-and-open decisional Diffie-Hellman} assumption for parameter $n$ is hard if for any ppt distinguisher $\D_1,\D_2$,
	\begin{eqnarray*}
		&&|\Pr[ \D_2(\mathsf{st},a_i,b_i,(\lb a_{j}b_{j}\rb)_{j\neq i})=1]\\
		&&-\Pr[ \D_2(\mathsf{st},a_i,b_i,(\lb c_{j}\rb)_{j\neq i})=1]|=\negl,
	\end{eqnarray*}
	where for $j\in[n]$, $a_j,b_j,c_j\leftarrow\Z_p$ and $(\mathsf{st}, i)\leftarrow\D_1(\lb 1\rb, (\lb a_j\rb,\lb b_j\rb)_{j\in [n]})$.
\end{definition}

\begin{lemma}
	Let the Choose-and-Open DDH assumption for parameter $n$ (\definitionref{def:CODDH}) hold over group $\G$. Then the One-round Diffie-Hellman based protocol  of \figureref{fig:KAtoOT} satisfies malicious receiver security (\definitionref{def:otSec}) in the UC model with respect to the 1-out-of-$n$ $\OOT^\E$ functionality.
\end{lemma}
\begin{proof}
The difference to the previous regimes is that now the simulator and sender will send their message before seeing the adversaries first message. The simulator for the malicious receiver is still straight forward. It sends the first message according to protocol. As previously, he will extract the receiver's input after seeing the malicious receveirs response $(r_i)_{i\in [n]}$. The input will be the index $i^*$ of random oracle $\H_{i^*}$ for which the malicious receiver makes the first query of the form $(g_1,\ldots,g_{i^*-1},g_{i^*+1},\ldots,g_n)$. The simulator sends the choice bit and all keys to the ideal functionality, where only the $i^*$th key will be computed according to the protocol, all other keys are uniformly random. 

If there is an environment $\D$ that can distinguish the real execution from the ideal world, then there is a distinguisher $\D'$ that breaks the CODDH assumption as follows. 

First, $\D'$ receives challenge $\lb 1\rb, (\lb a_j\rb,\lb b_j\rb)_{j\in [n]}$. He sends $(\lb a_j\rb)_{j\in [n]}$ to the malicious receiver as its OT message. 



	This proof follows the same structure as that of \lemmaref{lem:optKAtoOT} except as to how the receiver's chosen OT message $s_{\B, \ell}$ is defined. To compute $s_{\B, \ell}$ the simulator provides $\ell$ as the output of $\D_1$ and as such learns $b_\ell$ in the clear. The simulator can then locally compute $s_{\B, \ell}=b_\ell\lb x\rb$ where $\lb x\rb=\H(r_{\overline \ell})\oplus r_\ell$. 
	
	The other message $s_{\B,\overline \ell}$ is defined in the same way as in the proof of \lemmaref{lem:optKAtoOT}. In the real interaction the message has the form $s_{\B,{\overline \ell}}=\lb a_{\overline{\ell}}b_{\overline{\ell}}\rb \oplus \lb{(\delta_i+\gamma_j)b_{\overline{\ell}}}\rb$ and as such we can similarly show that it is uniformly distributed in the view of $\Adv$ by replacing $\lb a_{\overline{\ell}}b_{\overline{\ell}}\rb$ with $\lb c_{\overline{\ell}}\rb$.
\end{proof}

\begin{proof}
	
	Let the malicious receiver $\Adv$ make queries $q_1,...,q_Q$ to the random oracle $\H$ in this order. This then defines an $Q$-by-$Q$ matrix $A$ of ``interesting'' public keys where $A_{i,j}=\H(q_i)\oplus q_j$. Its possible that $\Adv$ did not query both $r_i$ values in which case the simulator will query $\H(r_1),\H(r_2)$ on their behalf and count these as $q_i$ queries.
	
	After receiving the $r_i$ values the simulator extracts the receiver's selection as $\ell$ s.t. $ q_j=r_{{\overline \ell}}, q_i=r_{\ell}$ and $j<i$ and $\overline \ell =3-\ell$. The simulator will then have two goals. 1) extract the receiver's choice message $s_{\B, \ell}=A_{j,i}^b$ and 2) use the random oracle to program $A_{i,j}^b$ to contain the SIDDH challenge $\lb c\rb$.
	
	
	Let us have the first phase of the SIDDH challenge $(\lb 1\rb, \lb b\rb)$. For all $\H(q_i)$ queries the simulator samples $a_i\gets \Z_p$ and implicitly defines $a_i=a+\delta_i$ and program $\H(q_i)=\lb a_i\rb =\lb a\rb \oplus \lb{\delta_i}\rb$. As such $A_{i,j}=\lb{a_i}\rb \oplus q_j =\lb a_i+\gamma_j\rb$ where $q_j =\lb\gamma_j\rb$. The distribution of $\H$ is unchanged by this modification.
	
	To extract the receiver's chosen message $s_{\B, \ell}$ the simulators provides the current state $\textsf{st}$ and $\lb x\rb=A_{j,i}$ as the output of $\D_1$. As input to $\D_2$ the simulator receives $(\lb a\rb, \lb xb\rb,\lb c^*\rb)$ where $c^*\in\{ab,c\}$. The receiver's chosen message is defined as $s_{\B, \ell}=\lb xb\rb$. 
	
	In the real interaction the final key agreement would be $s_{\B,{\overline \ell}}=A^b_{i,j}=b\lb a+\delta_i+\gamma_j\rb= \lb ab\rb \oplus \lb{(\delta_i+\gamma_j)b}\rb$. In the ideal interaction we substitute $\lb ab\rb$ with $\lb c\rb$ to obtain $s_{\B,{\overline \ell}}= \lb c\rb \oplus \lb{(\delta_i+\gamma_j)b}\rb$. Note that the simulator can not compute $s_{\B,{\overline \ell}}$ due to not knowing $\delta_i$ and $\gamma_j$. Moreover, $\delta_i$ is uniformly distributed in the view of $\Adv$ which implies that $(\delta_i+\gamma_j)b$ is independent of $c$ and therefore directly sampling  $s_{\B,{\overline \ell}}\gets \G$ yields the same distribution. Since sampling $s_{\B,{\overline \ell}}$ directly is the only change,  $\Adv$ distinguishing the sender's ideal/ideal output immediately gives a distinguisher for SIDDH.		
\end{proof}


\subsection{Optimized Diffie-Hellman based One-Round Endemic OT with UC Security }\label{sec:ddhProof}



\begin{definition}[Strong Interactive Decisional Diffie-Hellman (SIDDH) Assumption]\label{def:SIDDH}
	For a group $\G$, the \emph{strong interactive decisional Diffie-Hellman} assumption is hard if for any ppt distinguisher $\D_1,\D_2$,
	\begin{eqnarray*}
		&&|\Pr[ \D_2(\mathsf{st},\lb a\rb,\lb xb\rb,\lb ab\rb)=1]\\
		&&-\Pr[ \D_2(\mathsf{st},\lb a\rb,\lb xb\rb,\lb c\rb)=1]|=\negl,
	\end{eqnarray*}
	where $a\leftarrow\Z_p$, $b\leftarrow\Z_p$ and $c\leftarrow\Z_p$ and $(\mathsf{st},\lb x\rb)\leftarrow\D_1(\lb 1\rb,\lb b\rb)$.
\end{definition}

\begin{lemma}\label{lem:optKAtoOT}
	Let the Strong Interactive DDH assumption (\definitionref{def:SIDDH}) hold over group $\G$. Then the optimized Diffie-Hellman based protocol of \figureref{fig:optKAtoOT} satisfies malicious receiver security (\definitionref{def:otSec}) in the UC model with respect to the 1-out-of-2 $\OOT^\E$ functionality.
\end{lemma}
\begin{proof}
	
	Let the malicious receiver $\Adv$ make queries $q_1,...,q_Q$ to the random oracle $\H$ in this order. This then defines an $Q$-by-$Q$ matrix $A$ of ``interesting'' public keys where $A_{i,j}=\H(q_i)\oplus q_j$. Its possible that $\Adv$ did not query both $r_i$ values in which case the simulator will query $\H(r_1),\H(r_2)$ on their behalf and count these as $q_i$ queries.
	
	After receiving the $r_i$ values the simulator extracts the receiver's selection as $\ell$ s.t. $ q_j=r_{{\overline \ell}}, q_i=r_{\ell}$ and $j<i$ and $\overline \ell =3-\ell$. The simulator will then have two goals. 1) extract the receiver's choice message $s_{\B, \ell}=A_{j,i}^b$ and 2) use the random oracle to program $A_{i,j}^b$ to contain the SIDDH challenge $\lb c\rb$.
	
	
	Let us have the first phase of the SIDDH challenge $(\lb 1\rb, \lb b\rb)$. For all $\H(q_i)$ queries the simulator samples $a_i\gets \Z_p$ and implicitly defines $a_i=a+\delta_i$ and program $\H(q_i)=\lb a_i\rb =\lb a\rb \oplus \lb{\delta_i}\rb$. As such $A_{i,j}=\lb{a_i}\rb \oplus q_j =\lb a_i+\gamma_j\rb$ where $q_j =\lb\gamma_j\rb$. The distribution of $\H$ is unchanged by this modification.
	
	To extract the receiver's chosen message $s_{\B, \ell}$ the simulators provides the current state $\textsf{st}$ and $\lb x\rb=A_{j,i}$ as the output of $\D_1$. As input to $\D_2$ the simulator receives $(\lb a\rb, \lb xb\rb,\lb c^*\rb)$ where $c^*\in\{ab,c\}$. The receiver's chosen message is defined as $s_{\B, \ell}=\lb xb\rb$. 
	
	In the real interaction the final key agreement would be $s_{\B,{\overline \ell}}=A^b_{i,j}=b\lb a+\delta_i+\gamma_j\rb= \lb ab\rb \oplus \lb{(\delta_i+\gamma_j)b}\rb$. In the ideal interaction we substitute $\lb ab\rb$ with $\lb c\rb$ to obtain $s_{\B,{\overline \ell}}= \lb c\rb \oplus \lb{(\delta_i+\gamma_j)b}\rb$. Note that the simulator can not compute $s_{\B,{\overline \ell}}$ due to not knowing $\delta_i$ and $\gamma_j$. Moreover, $\delta_i$ is uniformly distributed in the view of $\Adv$ which implies that $(\delta_i+\gamma_j)b$ is independent of $c$ and therefore directly sampling  $s_{\B,{\overline \ell}}\gets \G$ yields the same distribution. Since sampling $s_{\B,{\overline \ell}}$ directly is the only change,  $\Adv$ distinguishing the sender's ideal/ideal output immediately gives a distinguisher for SIDDH.		
\end{proof}
