\section{Preliminaries}

\subsection{Notation}
$\sec$ denotes the security parameter. For $n\in\mathbb N$, $[n]:=\{1,\dots,n\}$ and $(s_j)_{j\neq\i}$ denotes ther ordered set  $(s_j)_{j\in[n]\setminus\{\i\}}$. We use $\Pi^{\A,\B}$, $\Pi$ when $\A$ and $\B$ are clear from the context,  to denote a protocol between two parties \A and \B. $\langle \A,\B\rangle$ denotes as the transcript of the protocol, which consists of all the messages sent between them. We use $(\A(a),\B(b))_{\Pi}$ to denote the joint output distribution of $\A$ and \B when interacting in protocol $\Pi$ with inputs $a$ and $b$.


\begin{definition}[Random Oracle]
A \emph{random oracle} over a set of domains and an image is a collection of functions \H that map an element $q$ within one of the domains  to a uniform element $\H(q)$ in the image.  
\end{definition}


\begin{definition}[Ideal Cipher]
An \emph{ideal cipher} over a set of tuples of domains and an image is a collection of functions $\pi$ such that for any element $\key$ of the first domain, $\pi_\key$ is a permutation that map an element $q$ within the second domain to a uniform element $\pi_{\key}(q)$ in the image.  
\end{definition}

\subsection{Key Agreement}

\begin{figure}
\centering
\framebox{
\begin{tikzpicture}
\node [anchor=west] at (\picsh,5) {\A:};
\node [anchor=west] at (5.5,5) {\B:};
\node [anchor=west] at (\picsh,4.5){$\mes_\A\leftarrow\A(\tape_\A)$};
\draw [thick, ->] (2.75,4.5)-- node [midway,above]{$\mes_\A$}(5.25,4.5);
\node [anchor=west] at (5.5,4){$\mes_\B\leftarrow\B(\tape_\B,\mes_\A)$};
\draw [thick, <-] (2.75,4)-- node [midway,above]{$\mes_\B$}(5.25,4);
\node [anchor=west] at (\picsh,3.5) {$\key_\A=\EK(\tape_\A,\mes_\B)$};
\node [anchor=west] at (5.5,3.5) {$\key_\B=\EK(\tape_\B,\mes_\A)$};
\end{tikzpicture}
}
\myvspace{-0.3cm}
\caption{The figure shows a key agreement protocol between parties \A and \B with random tapes $\tape_\A$ and $\tape_\B$. Correctness requires $\key_\A=\key_\B$.}
 \label{fig:UKA}
\end{figure}


\begin{definition}[(Two Message) Uniform Key Agreement (\UKA)] 
Let \G be a group.
We call a protocol $\Pi$ between two ppt parties \A and \B (two message) uniform key agreement if \A first sends a message $\mes_\A\in\G$ to \B and \B responds with a final message $\mes_\B$ and in the end, both establish a common key $\key$ (see \figureref{fig:UKA}) using a key establishing algorithm \EK. Further, we require three properties:
\begin{description}
\item[Correctness:]
$$
\Pr[\key_\A=\EK(\tape_\A,\mes_\B)=\EK(\tape_\B,\mes_\A)=\key_\B]\geq 1-\negl,
$$
where $\tape_\A\leftarrow\bits^*$, $\tape_\B\leftarrow\bits^*$, $\mes_\A\leftarrow\A(\tape_\A)$ and $\mes_\B\leftarrow\B(\tape_\B)$.
\item [Key-Indistinguishability:] For any ppt distinguisher \D and any polynomial size auxiliary input $z$,
$$ 
|\Pr[\D(z,\langle \A,\B\rangle, \key)=1]-\Pr[\D(z,\langle \A,\B\rangle, u)=1]|=\negl,
$$
where $\key$ is the established key between \A and \B and $u$ is a uniform element from the key domain.
\item [Uniformity:] For any ppt distinguisher \D and any polynomial size auxiliary input $z$, 
$$
|\Pr[\D(z,\mes_\A)=1]-\Pr[\D(z,u)=1]|=\negl,
$$
where $u$ is a uniform element from \G and $\mes_\A\leftarrow\A(\tape_\A)$.
\end{description}
When \A and \B can send their messages concurrently, we call it a one-round \UKA.
\end{definition}

\iffullversion
In \appendixref{sec:addKA},
\else
In the full version,
\fi
 we define multi-instance security notions when executing multiple instances of a key agreement. All the considered notions follow from the standard notions from above, but potentially with a polynomial security loss.


\subsection{Oblivious Transfer}



\begin{definition}[Ideal $k$-out-of-$n$ Oblivious Transfer]\label{def:ot}
An \emph{ideal $k$-out-of-$n$ oblivious transfer} is a functionality that interacts with two parties, a sender \send and a receiver \rec. \rec sends a set $\set\subseteq[n]$ of size $k$ to the functionality.

The functionality is publicly parameterized by one of the following message sampling methods:
\begin{description}
\item[] \textsc{Sender Chosen Message:} $\send$ sends the messages $(m_i)_{i\in[n]}$ to the functionality who sets $s_i:=m_i$.

\item[] \textsc{Receiver Chosen Message:} \rec sends the messages $(m_i)_{i\in[k]}$ to the functionality who sets $s_{\set_i}:=s'_i$ for $i\in[k]$ and uniformly samples $s_i\gets\{0,1\}^\ell$ for $i\in[n]\setminus\set$.

\item[] \textsc{Uniform Message:} The functionality uniformly samples $(s_1,...,s_n)\gets\{0,1\}^{\ell\times n}$. 
 
\item[] \textsc{Endemic:} If $\send$ is corrupt, $\send$ sends the messages $(s_i)_{i\in[n]}$ to the functionality. If \rec is corrupt, \rec sends the messages $(s'_i)_{i\in[n]}$ to the functionality who sets $s_{\set_i}:=s'_i$ for $i\in[k]$.
All remaining $s_i$ for $i\in [n]$ are uniformly sampled $s_i\gets\{0,1\}^\ell$ by  the functionality.
\end{description}

As specified by the message sampling method, the functionality constructs messages $(s_i)_{i\in[n]}$. Thereafter, the functionality sends $(s_i)_{i\in[n]}$ to \send and $(s_i)_{i\in \set}$ to \rec. We denote the ideal functionalities for sender chosen, receiver chosen, uniform message and endemic as $\OOT^\send, \OOT^\rec,\OOT^\U,\OOT^\E$, respectively. %
\end{definition}
\begin{remark}
We generalize this definition for the case where $n$ can be exponential. In addition, we consider the case when the set $\set$ is sampled uniformly by the functionality. We call this \emph{uniform selection} as opposed to \emph{receiver selection}. In this case, we denote the analogous oracles for $\OOT^\send, \OOT^\rec,\OOT^\U,\OOT^\E$ as $\OOT^{\send u},$ $\OOT^{\rec u}, \OOT^{\U u}, \OOT^{\E u}$, respectively. See \iffullversion \definitionref{def:got}.
\else
the full version for a precise definition.
\fi
\end{remark}


In our definition of OT, we use the simplified UC security definition that is sufficient for full UC security \cite{C:CanCohLin15}. 
\begin{definition}[$k$-out-of-$n$ Oblivious Transfer ($\OT_{k,n}$)]\label{def:otSec}
We call a protocol $\Pi$ between two ppt parties, a sender \send and a receiver \rec, a \emph{$k$-out-of-$n$ oblivious transfer} if %\rec has as input a set $\set\subset[n]$ of size $k$ and 
at the end, \send outputs $n$ strings $(s_i)_{i\in[n]}$ and \rec outputs $(s_i)_{i\in\set}$ and a set $\set\subset[n]$ s.t. $|\set|=k$. For security, we require two properties with respect to a functionality \OOT.
\begin{description}
\item[Security Against a Malicious Sender:] For any ppt adversary \Adv, there exists a ppt adversary \Adv' such that for any ppt environment \D and any polynomial size auxiliary input $z$
$$
|\Pr[\D(z,(\Adv,\rec)_{\Pi})=1] -\Pr[\D(z,(\Adv', \OOT))=1]|=\negl,
$$
where all algorithms receive input $1^\sec$. \rec additionally receives input \set.
\item[Security Against a Malicious Receiver:] For any ppt adversary \Adv, there exists a ppt adversary \Adv' such that for any ppt distinguisher \D and any polynomial size auxiliary input $z$
$$
|\Pr[\D(z, (\send, \Adv)_{\Pi})=1] -\Pr[\D(z, (\OOT,\Adv'))=1]|=\negl,
$$
where all algorithms receive input $1^\sec$.
\end{description}


We distinguish four different security notions.
\begin{description}
\item[Uniform Message Security:] The OT is secure with respect to $\OOT^\U$, i.e. $\OOT^\U$-secure.
\item[Sender Chosen Message Security:] The OT is secure with respect to $\OOT^{\send}$, i.e. $\OOT^{\send}$-secure.
\item[Receiver Chosen Message Security:] The OT is secure with respect to $\OOT^{\rec}$, i.e. $\OOT^{\rec}$-secure.
\item[Endemic Security:] The OT is secure with respect to $\OOT^{\E}$, i.e. $\OOT^{\E}$-secure.
\end{description}


\end{definition}

\begin{remark}
It is important to notice that we distinguish the functionality an OT provides from the ideal functionality for which the OT is secure. Here, 
$\OOT^\U$-security is the strongest security definition since a malicious party cannot tweak the distribution of the strings  $(s_i)_{i\in[n]}$. Endemic security gives the weakest security guarantees since in both cases, the malicious receiver and malicious sender case, the adversary can potentially choose the strings $(s_i)_{i\in\set}$.
\end{remark}

\begin{remark}
In the following, we assume that all messages from a sender or a receiver also contain a session identifier $\mathsf{sid}$ and that for every new session between a sender and a receiver, they receive access to a fresh random oracle that is unique to that session (local random oracle).  
\end{remark}

